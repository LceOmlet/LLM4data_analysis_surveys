\documentclass{article}


\usepackage{tikz}
\usetikzlibrary{calc,arrows,positioning}
\usepackage{xcolor}
\usepackage{pgfplots}

\begin{document}

  % 确保在导言区正确加载 xcolor 包
  \vspace{1cm} % 根据需要调整空间大小
  % 定义可调节的参数
  \newlength{\leftParentNodeWidth}
  \setlength{\leftParentNodeWidth}{3cm} % 左列父节点宽度,可调整
  
  \newlength{\rightParentNodeWidth}
  \setlength{\rightParentNodeWidth}{2cm} % 右列父节点宽度,可调整
  
  \newlength{\leftNodeWidth}
  \setlength{\leftNodeWidth}{2.8cm} % 左列子节点宽度,可调整
  
  \newlength{\middleNodeWidth}
  \setlength{\middleNodeWidth}{6cm} % 中间列节点宽度,可调整
  
  % 新增:定义中间节点的最小高度
  \newlength{\middleNodeHeight}
  \setlength{\middleNodeHeight}{0.2cm} % 中间列节点的最小高度,可调整
  
  \newlength{\rightNodeWidth}
  \setlength{\rightNodeWidth}{2.4cm} % 右列子节点宽度,可调整
  
  % 定义左、中、右列的垂直间距
  \newlength{\leftVerticalSpacing}
  \setlength{\leftVerticalSpacing}{0.9cm} % 左列垂直间距,可调整
  
  \newlength{\middleVerticalSpacing}
  \setlength{\middleVerticalSpacing}{0.6cm} % 中间列垂直间距,可调整
  
  \newlength{\rightVerticalSpacing}
  \setlength{\rightVerticalSpacing}{0.9cm} % 右列垂直间距,可调整
  
  % 定义列之间的水平间距
  \newlength{\horizontalSpacing}
  \setlength{\horizontalSpacing}{13.5cm} % 水平间距,可调整
  
  \begin{figure*}[h]

    \centering
\begin{tikzpicture}[
  auto,
  emptyNode/.style={
    draw=none,    % 不绘制边框
    fill=none,    % 不填充背景
    minimum height=0.8cm,  % 最小高度
    inner sep=0pt,   % 内部填充设置为 0,以确保节点内容紧凑
    align=center,
    scale=1.1
  },
  baseNode/.style={
    rectangle,
    draw,
    rounded corners,
    minimum height=0.8cm,
    inner sep=2pt,
    font=\small
  },
  leftParentNode/.style={
    emptyNode,
    text width=\leftParentNodeWidth,
    text=black,
    font=\bfseries,
  },
  rightParentNode/.style={
    emptyNode,
    text width=\rightParentNodeWidth,
    text=black,
    font=\bfseries
  },
  middleNode/.style={
    baseNode,
    text width=\middleNodeWidth,
    fill=gray!20,
    draw=gray!50,
    text=black,
    font=\tiny,
    minimum height=\middleNodeHeight
  },
  >=stealth'
]
 % 定义左列子章节的颜色
  \definecolor{DataManagementColor}{RGB}{31,119,180} % 蓝色
  \definecolor{ExploratoryDataAnalysisColor}{RGB}{44,160,44} % 绿色
  \definecolor{ImplementationColor}{RGB}{255,127,14} % 橙色
  \definecolor{AssessmentColor}{RGB}{148,103,189} % 紫色
  
  % 定义右列子章节的颜色
  \definecolor{DSLUnderstandingColor}{RGB}{214,39,40} % 红色
  \definecolor{DataQualityOptimizationColor}{RGB}{140,86,75} % 棕色
  \definecolor{AutoMLColor}{RGB}{23,190,207} % 青色
  \definecolor{AccessibilityColor}{RGB}{227,119,194} % 粉色
  
  % 定义左列父节点
  \node[leftParentNode] (left_parent) at (0,-2) {Task Solving};
  
  % 定义右列父节点
  \node[rightParentNode] (right_parent) at ($(left_parent)+(\horizontalSpacing,0)$) {Optimizations};
  
  % 中间参考节点起点(需要相应调整)
  \coordinate (middleStart) at ($(left_parent)!0.5!(right_parent)$);
  
  % 左列子节点
  \node[baseNode, fill=DataManagementColor!20, draw=DataManagementColor, text width=\leftNodeWidth, below=\leftVerticalSpacing of left_parent] (data_management) {Data Preparation};
  \node[baseNode, fill=ExploratoryDataAnalysisColor!20, draw=ExploratoryDataAnalysisColor, text width=\leftNodeWidth, below=\leftVerticalSpacing of data_management] (exploratory_data_analysis) {Exploratory DA};
  \node[baseNode, fill=ImplementationColor!20, draw=ImplementationColor, text width=\leftNodeWidth, below=\leftVerticalSpacing of exploratory_data_analysis] (implementing_methods) {Implementation};
  \node[baseNode, fill=AssessmentColor!20, draw=AssessmentColor, text width=\leftNodeWidth, below=\leftVerticalSpacing of implementing_methods] (assessing_results) {Assessment};
  
  % 右列子节点

  \node[baseNode, fill=DSLUnderstandingColor!20, draw=DSLUnderstandingColor, text width=\rightNodeWidth, below=\rightVerticalSpacing of right_parent] (dsl_understanding) {Reasoning};
  \node[baseNode, fill=DataQualityOptimizationColor!20, draw=DataQualityOptimizationColor, text width=\rightNodeWidth, below=\rightVerticalSpacing of dsl_understanding] (PFM_quality) {Data Quality};
  \node[baseNode, fill=AutoMLColor!20, draw=AutoMLColor, text width=\rightNodeWidth, below=\rightVerticalSpacing of PFM_quality] (automated_ml) {Automation};
  \node[baseNode, fill=AccessibilityColor!20, draw=AccessibilityColor, text width=\rightNodeWidth, below=\rightVerticalSpacing of automated_ml] (accessible_models) {Accessibility};
  
  % 定义中间节点的名称和对应的引用
  \def\middleNodes{
    {group1}/{\cite{022batini2009methodologies}, \cite{029dietterich1995overfitting}, \cite{037little2019statistical}, \cite{Bavishi2022NeurosymbolicRF}, \cite{Carrott2024CoqPytPN}, \cite{CatSQL}, \cite{DeletangRDCGMGW24}, \cite{FMcard}, \cite{FevrySFCK20}, \cite{HansenCF24}, \cite{JongZFSC22}, \cite{Khakhar2023PACPS}, \cite{Lehmann2023LanguageMA}, \cite{NLPTune}, \cite{NeuralDB}, \cite{TURL}, \cite{ahuja2023interventional}, \cite{blaauwbroekgraph2tac}, \cite{brand2023parameterized}, \cite{buchholzrobustness}, \cite{caruana2022data}, \cite{cummins2023large}, \cite{cummins2024meta}, \cite{feyposition}, \cite{garcia2010using}, \cite{grinsztajn2022tree}, \cite{gu2023few}, \cite{hongjin2024bright}, \cite{huang2023kosa}, \cite{lai2023ds}, \cite{li2023resdsql}, \cite{li2024flexkbqa}, \cite{ngom2024mallet}, \cite{popov2019neural}, \cite{vapnik2019rethinking}, \cite{vectorstorage}, \cite{vojivr2020editable}, \cite{wang2023large}, \cite{zhou2025depth}, \cite{zhu2024retrieval}},
    {group2}/{\cite{009brown2020language}, \cite{041zhai2022scaling}, \cite{JeonLLR24}, \cite{Ji2022SurveyOH}, \cite{Kaufmann2023ASO}, \cite{Lu2021FantasticallyOP}, \cite{Shi2023LargeLM}, \cite{Wang2023KnowledgeEF}, \cite{abbe2023generalization}, \cite{akyurek2022learning}, \cite{bender2021dangers}, \cite{chow2024performance}, \cite{codd2007relational}, \cite{dinh2024large}, \cite{gao2023retrieval}, \cite{khakhar2023pac}, \cite{liu2023jarvix}, \cite{morishita2023learning}, \cite{pei2023can}, \cite{salehin2024automl}, \cite{tan2024language}, \cite{vacareanu2024words}, \cite{wang2024context}, \cite{wu2024continual}, \cite{yang2024leandojo}},
    {group3}/{\cite{ActiveLearning}, \cite{DeepJoin}, \cite{Ditto}, \cite{Ember}, \cite{FMwrangling}, \cite{Starmie}, \cite{entityresolution}, \cite{fernandez2023large}, \cite{international2017dama}, \cite{jain2022jigsaw}, \cite{khoshafian1986object}, \cite{lyu2024automatic}, \cite{turing1936computable}, \cite{van2020data}, \cite{wang1996beyond}, \cite{zuo2022spine}},
    {group4}/{\cite{027davenport2017competing}, \cite{033kaelbling1996reinforcement}, \cite{040shmueli2011predictive}, \cite{DhuliawalaKXRLC24}, \cite{KamoiZZHZ24}, \cite{Rodriguez2024ExploringAO}, \cite{alshahwan2024automated}, \cite{hu2023survey}, \cite{schafer2023empirical}, \cite{schumann2013automated}, \cite{shao2023synthetic}, \cite{yang2024give}, \cite{yao2024tree}},
    {group5}/{\cite{094dai2017good}, \cite{097yoon2017semi}, \cite{102yang2018deep}, \cite{Atzeni2023InfusingLS}, \cite{Loem2023SAIEFS}, \cite{Miao2022LearningIA}, \cite{Neu2022GeneralizationBV}, \cite{du2024enhancing}, \cite{huamortizing}, \cite{li2024towards}, \cite{peng2022self}, \cite{weng2023g}},
    {group6}/{\cite{Dibia2023LIDAAT}, \cite{ReActTable}, \cite{dubiel2024device}, \cite{ko2024filling}, \cite{ma2023insightpilot}, \cite{nam2024using}, \cite{readyforNL2SQL}, \cite{smalllargemodelNL2SQL}, \cite{text2sqlevaluation}, \cite{zheng2024revolutionizing}},
    {group7}/{\cite{031guyon2003introduction}, \cite{KoziolekGHALE24}, \cite{Li2024IsPB}, \cite{jain2024r2e}, \cite{nam2024optimized}, \cite{repocomp}, \cite{singh2023augmenting}, \cite{yuan2023power}, \cite{zhengprise}},
    {group8}/{\cite{Pei2023CanLL}, \cite{Wang2024TheoremLlamaTG}, \cite{Zhang2024LargeLM}, \cite{gerussi2022llm}, \cite{jha2023counterexample}, \cite{truhn2023large}, \cite{vertsel2024hybrid}, \cite{ye2024satlm}},
    {group9}/{\cite{Ellis2020DreamCoderGG}, \cite{GPTuner}, \cite{KGobjectrecognition}, \cite{Tang2024WorldCoderAM}, \cite{constructKG}, \cite{deem}, \cite{qi2024cleanagent}},
    {group10}/{\cite{ZhangS24}, \cite{heer2010declarative}, \cite{kim2022cicero}, \cite{myatt2007making}, \cite{shih2018declarative}},
    {group11}/{\cite{020arlot2010survey}, \cite{032hastie2009elements}, \cite{HuZWCM0WSXZCY0K24}, \cite{WangCY0L0J24}},
    {group12}/{\cite{EoTGE}, \cite{HsuMTW23}, \cite{SongY00024}, \cite{sayed2024gizaml}},
    {group13}/{\cite{Hollmann2023LargeLM}, \cite{LinDJZ24}, \cite{malberg2024felix}, \cite{sordoni2024joint}},
    {group14}/{\cite{ACSV}, \cite{YanWWML24}, \cite{huh2023pool}},
    {group15}/{\cite{LLMschema}, \cite{ni2024iterclean}, \cite{vos2022towards}},
    {group16}/{\cite{li2024table}, \cite{tabulargeneration}, \cite{zhang2024jellyfish}},
    {group17}/{\cite{CHORUS}, \cite{Nobari2023DTTAE}},
    {group18}/{\cite{Wang2023SoloDD}},
    {group19}/{\cite{JiangSQLZZY25}},
    {group20}/{\cite{bai2024transformers}},
    {group21}/{\cite{li2024can}},
    {group22}/{\cite{ZeroED}},
    {group23}/{\cite{cheng2022binding}},
    {group24}/{\cite{reizingerposition}},
    {group25}/{\cite{GrandWBOLTA24}}
}
  
  % 放置中间列节点
  \def\yShift{-1cm} % 累积垂直偏移量
  
  \foreach \name/\text in \middleNodes {
    \node[middleNode] (\name) at ($(middleStart)-(0,\yShift+\leftVerticalSpacing)$) {\text};
    \pgfmathparse{\yShift+\middleVerticalSpacing}
    \xdef\yShift{\pgfmathresult}
  }
  
  % Generated connections based on co-occurrence patterns
  \draw[->, thin, draw=DSLUnderstandingColor] (group1.east) to [out=0, in=180] (dsl_understanding.west);
  \draw[->, thin, draw=AutoMLColor] (group2.east) to [out=0, in=180] (automated_ml.west);
  \draw[->, thin, draw=DataManagementColor] (data_management.east) to [out=0, in=180] (group3.west);
  \draw[->, thin, draw=AssessmentColor] (assessing_results.east) to [out=0, in=180] (group4.west);
  \draw[->, thin, draw=DataQualityOptimizationColor] (group5.east) to [out=0, in=180] (PFM_quality.west);
  \draw[->, thin, draw=ExploratoryDataAnalysisColor] (exploratory_data_analysis.east) to [out=0, in=180] (group6.west);
  \draw[->, thin, draw=DSLUnderstandingColor] (group6.east) to [out=0, in=180] (dsl_understanding.west);
  \draw[->, thin, draw=ImplementationColor] (implementing_methods.east) to [out=0, in=180] (group7.west);
  \draw[->, thin, draw=DSLUnderstandingColor] (group7.east) to [out=0, in=180] (dsl_understanding.west);
  \draw[->, thin, draw=AssessmentColor] (assessing_results.east) to [out=0, in=180] (group8.west);
  \draw[->, thin, draw=DSLUnderstandingColor] (group8.east) to [out=0, in=180] (dsl_understanding.west);
  \draw[->, thin, draw=DataManagementColor] (data_management.east) to [out=0, in=180] (group9.west);
  \draw[->, thin, draw=DSLUnderstandingColor] (group9.east) to [out=0, in=180] (dsl_understanding.west);
  \draw[->, thin, draw=ExploratoryDataAnalysisColor] (exploratory_data_analysis.east) to [out=0, in=180] (group10.west);
  \draw[->, thin, draw=ImplementationColor] (implementing_methods.east) to [out=0, in=180] (group11.west);
  \draw[->, thin, draw=ImplementationColor] (implementing_methods.east) to [out=0, in=180] (group12.west);
  \draw[->, thin, draw=AutoMLColor] (group12.east) to [out=0, in=180] (automated_ml.west);
  \draw[->, thin, draw=DataManagementColor] (data_management.east) to [out=0, in=180] (group13.west);
  \draw[->, thin, draw=AutoMLColor] (group13.east) to [out=0, in=180] (automated_ml.west);
  \draw[->, thin, draw=DataManagementColor] (data_management.east) to [out=0, in=180] (group14.west);
  \draw[->, thin, draw=DataQualityOptimizationColor] (group14.east) to [out=0, in=180] (PFM_quality.west);
  \draw[->, thin, draw=AutoMLColor] (group14.east) to [out=0, in=180] (automated_ml.west);
  \draw[->, thin, draw=DataManagementColor] (data_management.east) to [out=0, in=180] (group15.west);
  \draw[->, thin, draw=DSLUnderstandingColor] (group15.east) to [out=0, in=180] (dsl_understanding.west);
  \draw[->, thin, draw=DataQualityOptimizationColor] (group15.east) to [out=0, in=180] (PFM_quality.west);
  \draw[->, thin, draw=DataManagementColor] (data_management.east) to [out=0, in=180] (group16.west);
  \draw[->, thin, draw=DataQualityOptimizationColor] (group16.east) to [out=0, in=180] (PFM_quality.west);
  \draw[->, thin, draw=DataManagementColor] (data_management.east) to [out=0, in=180] (group17.west);
  \draw[->, thin, draw=ExploratoryDataAnalysisColor] (exploratory_data_analysis.east) to [out=0, in=180] (group17.west);
  \draw[->, thin, draw=DataQualityOptimizationColor] (group17.east) to [out=0, in=180] (PFM_quality.west);
  \draw[->, thin, draw=ImplementationColor] (implementing_methods.east) to [out=0, in=180] (group18.west);
  \draw[->, thin, draw=DataManagementColor] (data_management.east) to [out=0, in=180] (group18.west);
  \draw[->, thin, draw=ExploratoryDataAnalysisColor] (exploratory_data_analysis.east) to [out=0, in=180] (group18.west);
  \draw[->, thin, draw=DataQualityOptimizationColor] (group18.east) to [out=0, in=180] (PFM_quality.west);
  \draw[->, thin, draw=ExploratoryDataAnalysisColor] (exploratory_data_analysis.east) to [out=0, in=180] (group19.west);
  \draw[->, thin, draw=AutoMLColor] (group19.east) to [out=0, in=180] (automated_ml.west);
  \draw[->, thin, draw=ImplementationColor] (implementing_methods.east) to [out=0, in=180] (group20.west);
  \draw[->, thin, draw=DSLUnderstandingColor] (group20.east) to [out=0, in=180] (dsl_understanding.west);
  \draw[->, thin, draw=AutoMLColor] (group20.east) to [out=0, in=180] (automated_ml.west);
  \draw[->, thin, draw=ExploratoryDataAnalysisColor] (exploratory_data_analysis.east) to [out=0, in=180] (group21.west);
  \draw[->, thin, draw=DSLUnderstandingColor] (group21.east) to [out=0, in=180] (dsl_understanding.west);
  \draw[->, thin, draw=AutoMLColor] (group21.east) to [out=0, in=180] (automated_ml.west);
  \draw[->, thin, draw=DataManagementColor] (data_management.east) to [out=0, in=180] (group22.west);
  \draw[->, thin, draw=DSLUnderstandingColor] (group22.east) to [out=0, in=180] (dsl_understanding.west);
  \draw[->, thin, draw=DataQualityOptimizationColor] (group22.east) to [out=0, in=180] (PFM_quality.west);
  \draw[->, thin, draw=AutoMLColor] (group22.east) to [out=0, in=180] (automated_ml.west);
  \draw[->, thin, draw=DSLUnderstandingColor] (group23.east) to [out=0, in=180] (dsl_understanding.west);
  \draw[->, thin, draw=AutoMLColor] (group23.east) to [out=0, in=180] (automated_ml.west);
  \draw[->, thin, draw=AssessmentColor] (assessing_results.east) to [out=0, in=180] (group24.west);
  \draw[->, thin, draw=DSLUnderstandingColor] (group24.east) to [out=0, in=180] (dsl_understanding.west);
  \draw[->, thin, draw=AutoMLColor] (group24.east) to [out=0, in=180] (automated_ml.west);
  \draw[->, thin, draw=ImplementationColor] (implementing_methods.east) to [out=0, in=180] (group25.west);
  \draw[->, thin, draw=ExploratoryDataAnalysisColor] (exploratory_data_analysis.east) to [out=0, in=180] (group25.west);
  \draw[->, thin, draw=DSLUnderstandingColor] (group25.east) to [out=0, in=180] (dsl_understanding.west);

  
  \end{tikzpicture}
  \caption{\textbf{An Overview of How PFMs Improve Data Analysis. Updated with 8-subsection co-occurrence based grouping.} The figure illustrates the relationships between key data analysis tasks (left column)—Data Management, Exploratory Data Analysis, Implementation, and Assessment—and the Improvements enabled by PFMs (right column). The middle column lists representative methods and studies that bridge specific tasks to the corresponding improvements. Arrows indicate how PFMs address challenges within each task, facilitating the transition towards optimized data analysis processes. This visual framework underscores the multifaceted role of PFMs in systematically enhancing data analysis by connecting practical tasks with advanced improved strategies and the core role of scaling and consolidating domain-specific language (DSL) during data analysis.}

  \end{figure*}
\end{document}