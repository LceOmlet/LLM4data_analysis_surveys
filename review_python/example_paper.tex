%%%%%%%% ICML 2025 EXAMPLE LATEX SUBMISSION FILE %%%%%%%%%%%%%%%%%

\documentclass{article}

% Recommended, but optional, packages for figures and better typesetting:
\usepackage{microtype}
\usepackage{graphicx}
\usepackage{subfigure}
\usepackage{booktabs} % for professional tables

% hyperref makes hyperlinks in the resulting PDF.
% If your build breaks (sometimes temporarily if a hyperlink spans a page)
% please comment out the following usepackage line and replace
% \usepackage{icml2025} with \usepackage[nohyperref]{icml2025} above.
\usepackage{hyperref}


% Attempt to make hyperref and algorithmic work together better:
\newcommand{\theHalgorithm}{\arabic{algorithm}}

% Use the following line for the initial blind version submitted for review:
\usepackage{icml2025}

% If accepted, instead use the following line for the camera-ready submission:
% \usepackage[accepted]{icml2025}

% For theorems and such
\usepackage{amsmath}
\usepackage{amssymb}
\usepackage{mathtools}
\usepackage{amsthm}

\usepackage{array}
\usepackage{multirow}
\usepackage{arydshln}
\usepackage{makecell}
% \usepackage[linesnumbered,ruled,vlined]{algorithm2e}
\usepackage{amsthm}
\newcommand{\method}[2]{\hyperlink{cite.#2}{#1}}
\usepackage{lipsum} % Just for sample text

% Modify the footnote command to not show the mark
\makeatletter
\def\blfootnote{\xdef\@thefnmark{}\@footnotetext}
\makeatother

\newtheorem{theorem}{\textbf{Theorem}} % Define a new theorem environment named "theorem"
\newtheorem{remark}{Remark} % Define a new theorem environment named "theorem"
\newtheorem{lemma}{Lemma} % Define a new theorem environment named "theorem"
\newtheorem{Proposition}{Proposition}
\newtheorem{definition}{Definition}
\newtheorem{researchq}{RQ}
\newtheorem{property}{property}


% \usepackage{tocbasic}% One more KOMA-Script package added.
% Using tocbasic to make the table of contents:
% \renewcommand*{\tableofcontents}{\listoftoc[\contentsname]{toc}}
% \DeclareTOCStyleEntries[raggedentrytext]{tocline}{section,subsection,subsubsection,paragraph,subparagraph}

% \usepackage{tocbibind}    % 如果希望将目录显示在文档中
% \setcounter{secnumdepth}{0}

% if you use cleveref..
\usepackage[capitalize,noabbrev]{cleveref}

%%%%%%%%%%%%%%%%%%%%%%%%%%%%%%%%
% THEOREMS
%%%%%%%%%%%%%%%%%%%%%%%%%%%%%%%%


% Todonotes is useful during development; simply uncomment the next line
%    and comment out the line below the next line to turn off comments
%\usepackage[disable,textsize=tiny]{todonotes}
\usepackage[textsize=tiny]{todonotes}


% The \icmltitle you define below is probably too long as a header.
% Therefore, a short form for the running title is supplied here:
\icmltitlerunning{Submission and Formatting Instructions for ICML 2025}

\begin{document}

\twocolumn[
\icmltitle{Unsupervised Multi-modal Feature Alignment \\ for Time Series Representation Learning}

% It is OKAY to include author information, even for blind
% submissions: the style file will automatically remove it for you
% unless you've provided the [accepted] option to the icml2025
% package.

% List of affiliations: The first argument should be a (short)
% identifier you will use later to specify author affiliations
% Academic affiliations should list Department, University, City, Region, Country
% Industry affiliations should list Company, City, Region, Country

% You can specify symbols, otherwise they are numbered in order.
% Ideally, you should not use this facility. Affiliations will be numbered
% in order of appearance and this is the preferred way.
\icmlsetsymbol{equal}{*}

\begin{icmlauthorlist}
\icmlauthor{Firstname1 Lastname1}{equal,yyy}
\icmlauthor{Firstname2 Lastname2}{equal,yyy,comp}
\icmlauthor{Firstname3 Lastname3}{comp}
\icmlauthor{Firstname4 Lastname4}{sch}
\icmlauthor{Firstname5 Lastname5}{yyy}
\icmlauthor{Firstname6 Lastname6}{sch,yyy,comp}
\icmlauthor{Firstname7 Lastname7}{comp}
%\icmlauthor{}{sch}
\icmlauthor{Firstname8 Lastname8}{sch}
\icmlauthor{Firstname8 Lastname8}{yyy,comp}
%\icmlauthor{}{sch}
%\icmlauthor{}{sch}
\end{icmlauthorlist}

\icmlaffiliation{yyy}{Department of XXX, University of YYY, Location, Country}
\icmlaffiliation{comp}{Company Name, Location, Country}
\icmlaffiliation{sch}{School of ZZZ, Institute of WWW, Location, Country}

\icmlcorrespondingauthor{Firstname1 Lastname1}{first1.last1@xxx.edu}
\icmlcorrespondingauthor{Firstname2 Lastname2}{first2.last2@www.uk}

% You may provide any keywords that you
% find helpful for describing your paper; these are used to populate
% the "keywords" metadata in the PDF but will not be shown in the document
\icmlkeywords{Machine Learning, ICML}

\vskip 0.3in
]

% this must go after the closing bracket ] following \twocolumn[ ...

% This command actually creates the footnote in the first column
% listing the affiliations and the copyright notice.
% The command takes one argument, which is text to display at the start of the footnote.
% The \icmlEqualContribution command is standard text for equal contribution.
% Remove it (just {}) if you do not need this facility.

%\printAffiliationsAndNotice{}  % leave blank if no need to mention equal contribution
\printAffiliationsAndNotice{\icmlEqualContribution} % otherwise use the standard text.

\begin{abstract}
Unsupervised representation learning (URL) for time series data has garnered significant interest due to its remarkable adaptability across diverse downstream applications. It is challenging to ensure the utility for downstream tasks by only identifying patterns from the temporal domain features since the goal of URL differs from supervised learning methods. A variety of time series transform techniques, e.g., spectral features, wavelet transformed features, features in image form and symbolic features, etc., transform time series to multi-modal informative features. This study proposes an alignment-based method to harvest transform properties from them. In contrast to common methods that fuse features from multiple modalities, our proposed method simplifies the final neural architecture by retaining a single time series encoder. By validating the proposed method on a diverse set of time series datasets, our approach outperforms existing state-of-the-art URL methods across 3 diverse downstream tasks. This paper introduces a novel model-agnostic paradigm for time series URL, paving a new research direction.
\end{abstract}




\section{Introduction}

Time series describes a variable over time encompassing monitoring measures such as an organism's electrical signals and the CPU usage of a system or device. Multivariate time series (MTS) involves sets of dependent variables, playing a pivotal role across domains from finance to healthcare and the natural sciences~\cite{bennett2022detection, kim2019fault,riley2021internet, bi2023accurate}.


Time series pose complexities in pattern identification and analysis. This further increases the difficulty of obtaining labels for supervised learning~\cite{tonekaboni2021unsupervised}. Thus, unsupervised representation learning (URL) for time series~\cite{eldele2021time, yang2022unsupervised, liang2023contrastive,yue2022ts2vec} has emerged. \textit{URL aims to train a neural network (called encoder), without requiring labels, to encode the data into geometric signatures (or embedding), to capture inherent patterns of the raw data.} The learned representation, necessitating minimal annotation~\cite{zerveas2021transformer}, proves valuable in a variety of downstream tasks including temporal data classification, clustering~\cite{paparrizos2023odyssey, bonifati2022time2feat}, anomaly detection~\cite{liang2023contrastive, yue2022ts2vec, schmidl2022anomaly}, retrieval~\cite{zhu2020deep, paparrizos2019grail}, etc. Moreover, akin to deep clustering~\cite{guo2017improved} and manifold learning~\cite{balasubramanian2002isomap} methods, URL can help the encoders better capture local~\cite{pan2014evaluation,he2003locality} or global~\cite{batista2011complexity} structures, which can be induced from patterns present in the original samples and learn from prior deductive knowledge. As a result, unsupervised methods with simple classification heads on learned representations may outperform methods tailored for downstream tasks~\cite{liang2023contrastive}.

Researchers have explored transforming time series data into image~\cite{wang2015encoding,park2023meta} or symbol bags~\cite{schafer2012sfa, tang2020interpretable} to comprehend them better. Various feature engineering methods are proposed to parse time series and expose abundant informative patterns that are more tangible to identify. However, the challenges for leveraging these transforms can be further explored as follows. %\footnote{the reasons of challenge are not significant}

\blfootnote{Code available at \href{https://anonymous.4open.science/r/MMFA-8E8B}{https://anonymous.4open.science/r/MMFA-8E8B}.}
%compared to computer vision (CV)~\cite{chen2020simple, li2020prototypical, li2022univip} and natural language processing (NLP)~\cite{gao2021simcse, zhang2022unsupervised}, time series data poses greater feature engineering challenges for representative learning. In the CV and NLP domain, large cardinalities of datasets or inherent redundancy for discriminative patterns facilitate data conversion to URL-appropriate inputs. 

%However,  Researchers have proposed 
 %\footnote{what is the meaning?}

\paragraph{Disjoint transform properties of multi-modal features} Invariance refers to representation unchanged when input changes, while equivariance refers to representation changing with input, according to certain rules~\cite{guto2023icml}. They are transforming properties of the transforms and encoders. These transform properties can introduce prior knowledge to the inductive bias to unsupervised learning. Since the variabilities of patterns learned from each feature differ, learning from them individually introduces gaps between learned transform properties~\cite{yang2022unsupervised}. Besides, when fusing knowledge learned from the features together, the encoder merely needs to identify easily learnable patterns~\cite{ren2022co}, without considering algorithmic associations between different multi-modal patterns, which should be learned with priority, other than learning from shortcuts and blended blind source noises. This can lead to the learning of false patterns~\cite{robinson2021can}, resulting in information loss~\cite{le1988preservation}, data misinterpretation~\cite{geirhos2020shortcut}, and decreased model performance~\cite{puli2023don}.
 

\paragraph{Coupled transforms and encoders during inference} Feature fusion structures may overly complicate the neural network~\cite{yang2022unsupervised,tang2020interpretable, park2023meta}. Most feature engineering methods for time series are highly coupled with the neural structure of the encoder. Moreover, redundant computational consumption for such feature acquirement during inference is inevitable. For example, a dual tower structure~\cite{yang2022unsupervised} fuses time and spectral domain features, filling the inductive gap between these views. It is necessary to acquire Fourier-transformed features first and encode both features whenever we use the encoder. The cost would be unbearable if more time-consuming transforms and encoders like continuous wavelet transform~\cite{grossmann1990reading} and a 2D ResNet architecture~\cite{wang2015encoding} were involved during inference.


To overcome the above challenges, we inject prior inductive bias~\cite{hu2021model} implemented by multi-modal transforms into the raw time series encoder and propose a model-agnostic framework. This encoder recovers the most salient local and global arrangement preserved by various transforms and feature extraction processes. 

%Thus, we construct graphs that depict the alignment of the feature entities. The graphs are naturally composed of subgraphs, on which encoders learn local arrangement individually, and associations of crucial patterns implied globally among multi-modal features. %\footnote{what is the relationship to the previous sentence}

We take advantage of a regularization method that treats the encoders as a graph spectral embedding learner on the graph which aligns the multi-modal features. The alignment with the transforms fine-tunes the global arrangement of the representations, which originally preserves only the local arrangement of the raw time series.
%by connecting it to the other subgraph. 
Therefore, only preserving the neural encoder for the temporal domain is adequate during inference. Our analysis demonstrates that this framework significantly outperforms state-of-the-art methods on multiple downstream tasks.

%Notably, the method of feature alignment remains underexplored in URL for time series data, making this work the first to explore the alignment's potential in URL for time series and achieve performance beyond state-of-art approaches. We also analyze the effect of our training objective which aligns representations of multi-modal features and neural feature extractors theoretically.


%These graphs can then be aligned together to create a super-graph and modify the initial connectivity between nodes in each of the sub-graphs. This alignment process serves to 

% Furthermore, to learn equivalent multi-modal patterns from the graph, we propose regularization methods that treat the encoders as a spectral embedding learner on the graph. In practice, we leverage compositions of informative transforms and neural encoders to recover the data alignment represented by the graph. During the training phase, the encoders gradually search the eigenfunctions of the graph laplacian operator with small eigenvalues. We recover more meaningful clusters and acquire an encoder that learns the related patterns introduced by transforms. As a result, our multi-modal feature alignment framework (MMFA) outperforms existing state-of-the-art approaches. % \footnote{what?}%By combining feature alignment and regularization methods, we introduce a novel URL framework that outperforms existing state-of-the-art URL approaches.

The main contributions of this paper are summarized as follows:

\begin{itemize}
\item We introduce a novel feature-aligned URL framework, leveraging intricate feature engineering and signal processing methods, outperforming currently available methods while preserving high scalability.%}\footnote{too weak}

\item The proposed method learns local and global arrangement of multi-modal features, and sufficiently leverages different inductive biases introduced.
%%这个和上一条contribution没关系,应当强调贡献

\item The proposed time series URL paradigm based on alignment and regularization, points toward a new model-agnostic and theoretically-oriented URL for TS research direction.

\item Through experiments on time series datasets in different domains and with different characteristics (i.e., length, dimensionality, and train set sizes), we find that our proposed approach outperforms all existing URL approaches by a large margin, comparable (or even better than) to approaches tailored to multiple downstream tasks.
\end{itemize}


\begin{figure}[t]
\centerline{\includegraphics[width=\linewidth]{overview.pdf}}
\caption{\textbf{An overview of the MMFA framework.} A range of transforms is applied to raw time series to produce informative features across multiple modalities. These features are subsequently processed by neural feature extractors to identify different patterns. Following this, the representations are mapped to an embedding space and aligned via the regularization method.}
\label{overview}
\end{figure}



\section{Preliminaries}


This section defines the key concepts used in the paper. First, we clarify the data type we study in this paper, multivariate time series. Then we introduce the transforms and encoders with their domain and codomain.

\begin{table}
    \centering
    \caption{Composition of Transform Operators and Neural Feature Extractors.}\label{tab:TFpairs}
    \begin{tabular}{p{0.11\textwidth}c:p{0.15\textwidth}cp{0.8\textwidth}cp{0.18\textwidth}}
        \toprule
        \multicolumn{2}{l:} {\textbf{Transform Operator}}& \multicolumn{2}{c} {\textbf{Feature Extractor}} \\
        \midrule
        \multicolumn{2}{l:} {Raw Time Series} & \multicolumn{2}{c}{ST} \\
        \hdashline
        \multicolumn{2}{l:}{FFT} & \multicolumn{2}{c}{ResNet1d} \\
        \hdashline
        \multirow{2}{*}{Image Encoding} & PR & \multicolumn{2}{c}{\multirow{5}{*}{ResNet12}}  \\
        \cdashline{2-2}
        & GADF \\
        \cdashline{1-2}
        \multirow{3}{*}{CWT} & db1   \\ 
        \cdashline{2-2}
        & dmey  \\ 
        \cdashline{2-2}
        & coif5   \\
        \cdashline{1-4}
        \multicolumn{2}{l}{\multirow{2}{*}{SFA}} & \multirow{2}{*}{Longformer} & pre-trained \\
        \cdashline{4-4}
        & & & Random \\
        \bottomrule
    \end{tabular}
\end{table}
\subsection{Unsupervised Representation Learning for Multivariate Time Series} 

Multivariate time series is a set of variables, containing observations ordered by successive time. We denote a MTS sample with $D$ variables (a.k.a. dimensions or channels) and $T$ timestamps (a.k.a. length) as $x \in \mathbb{R}^{D \times T}$, and a dataset containing $N$ samples as the $X = \{x_1, x_2, \dots ,x_N\}\in\mathbb{R}^{N\times D \times T}$. 

Given an MTS dataset $X$, the goal of unsupervised representation learning (URL) is to train a neural network model (encoder) $f:\mathbb{R}^{D\times T} \to \mathbb{R}^{d_z}$, where $d_z$ is the dimensionality of the representation, the acquired representation $z_i = f(x_i)$ can be informative for downstream tasks, e.g., classification and forecasting. ``Unsupervised'' means that the labels of downstream tasks are unavailable when training $f$. To simplify the notation, we denote $f(X) = \{z_1, z_2, \dots, z_N\} \in \mathbb{R}^{N \times d_z}$ in following sections. 

\subsection{Neural Encoders and Corresponded Input Features}\label{sec:neur_encoder_main}

Concerning raw MTS data and multi-modal features transformed, we denote the features as $x^{(i)} = T^{(i)}(x)$ in this work, and $i$ denotes the index of the selected transform. The transformed features are highly heterogeneous from the original raw MTS data. Therefore, we designed different neural architectures for the feature encoders w.r.t the multi-modal features, $z^{(i)} = f^{(i)}(x^{(i)})$. They can be composited as $f^{(i)} \circ T^{(i)}$. But the selection of $f^{(i)}$ and $T^{(i)}$ should be compatible. The compositions of them are listed in \textbf{Tab.}~\ref{tab:TFpairs}. We introduce the most important encoder for the raw time series below, and the details about others are shown in the appendix, \textbf{Sec.}~\ref{sec:neur_encoder}.

\paragraph{Shapelet Learning Neural Network}

In our implementation of \textbf{MMFA} we choose shapelet transformer (ST)~\cite{liang2023contrastive} for the backbone encoder finally gained after unsupervised learning. ST transforms the time series signal into energy sequences along with the time axis, using a learned wavelet-like small sequence, shapelet, sliding from one side of the signal to another side and outputs distances from the small sequence to the subsequences of the signal. Lengths of shapelets vary just like dilating or compressing wavelets to different scales. This allows ST to fit the signal shapes of interest adaptively and automatically search for an optimal temporal-spectral resolution trade, similar to CWT. We use ST in this work as the raw temporal data feature extractor, $f: \mathbb{R}^{D \times T} \to \mathbb{R}^{d_z}$. It takes the output of identity transform of raw time series, $T^{(id)}: \mathbb{R}^{D\times T} \to \mathbb{R}^{D \times T}$. Other encoders are defined in a similar way in \textbf{Sec.}~\ref{sec:transforms}.

For the sake of simplicity, we denote \( X^{(i)} = T^{(i)}(X)  = \{ x_1^{(i)}, x_2^{(i)}, \dots , x_N^{(i)} \} \in \mathbb{R}^{N \times \text{dim} T^{(i)}} \) and $ Z^{(i)} = f^{(i)}(X^{(i)}) =  \{ z^{(i)}_1, z^{(i)}_2, \dots , z^{(i)}_N\}, i \in [0,k]$ in subsequent sections, where $T^{(i)}$, $f^{(i)}$ are $i$th in $k$ predefined transform and encoder involved and $f^{(0)} = \text{Id}$ is an identity function. Eventually, we discuss the alignment graph and eigenfunctions defined on the sample space, which encompasses not only raw data but also transformed features. The domain of the eigenfunctions can be denoted as \( \mathcal{S}^{\text{\textit{data}}} = \sqcup_{i \in [0,k]} \text{Im}(T^{(i)}) \), given $k$ transforms, where $\text{Im}(T^{(i)})$ denotes images of the transforms which is the space transformed features lies in.


\section{Overview}\label{sec:over}

In this section, we overview the framework of the proposed multi-modal feature alignment (MMFA). Generally, we treat the transformed features as individual entities in the sample space $\mathcal{S}^{\text{\textit{data}}} = \sqcup_{i \in [0,k]} \text{Im}(T^{(i)})$. This procedure maps the euclidean space $\mathcal{S}^{\text{\textit{data}}}$ to an generic embedding space $\mathbb{R}^{d_z}$ through a piecewise function $\phi: \mathcal{S}^{\text{\textit{data}}} \to \mathbb{R}^{d_z}$. It can be defined as a piecewise function. 


\begin{equation}
\begin{aligned}
        \forall i\in[0,k], \forall x \in \text{Im}(T^{(i)}) & \ \ \ \ \ \phi(x) = f^{(i)}(x)
\end{aligned}
\end{equation}

For learning $\phi$ with high utility, leveraging spectral embedding theory, URL relies on an undirected graph $\mathcal{G}(\mathcal{X}, w)$ that indicates sample arrangement lies in the manifold in the original high dimensional space from where raw data is collected, where $\mathcal{X}\in [(k + 1)N]$ denotes the subscript for raw MTS and the transformed features. In this paper, the relation matrix $w \in \mathbb{R}^{(k + 1)N \times (k + 1)N}$ is constructed by connecting entities transformed from the same time series, and similar patterns lie in each of the feature modalities. 

\begin{equation}\label{eq:graph_weight}
w_{ij} =:   \begin{cases}  A_{T_{(\mathcal{X}_i)}, T_{(\mathcal{X}_j)}}(\mathcal{X}_i, \mathcal{X}_j) , i \neq j \\
 0, i = j 
\end{cases}
\end{equation}


\begin{figure*}
    \centering
    \includegraphics[width=\textwidth]{contrast.pdf}
    \caption{\textbf{Two types of patterns of interest with three multi-modal feature views.} (raw time series, CWT, and FFT of time series) With a green dash circle indicating patterns that are hard to distinguish and, a red dash circle indicating patterns that are easy to identify. The feature encoders take these multi-modal features as input. Difficulties for them to capture certain patterns vary with different transforms, causing different probabilities for them in determining two sample points to be identical.}
    \label{fig:contrast}
\end{figure*}

Eq~\ref{eq:graph_weight} describes a sub-graph of global alignment,  $\mathcal{G}$, for approximate reconstruction using low-dimension representations. The adjacency function $A_{T_{(\mathcal{X}_i)}, T_{(\mathcal{X}_j)}}$ describes whether the data points, $\mathcal{X}_i$ and $ \mathcal{X}_j$, are transformed from the same sample. This relation strictly suffices. They share equivalent patterns with different modalities generated by transform operators $T_{(\mathcal{X}_i)}$ and $T_{(\mathcal{X}_j)}$ jointly, where $T_{(\mathcal{X}_i)}$ denotes the transform used to acquire the entity $\mathcal{X}_i$. We further describe two kinds of edges of the graph and the corresponding mechanisms about how they contribute to the final utility of the representation in \textbf{Sec.}~\ref{sec:desi}. An intuitive example of how transform affects the weights globally of the locally aligned graphs is shown in \textbf{Fig.}~\ref{fig:contrast}. For the extracted feature vectors $\mathcal{Z} \in \mathbb{R}^{(k+1)N \times d_z}$, which is stacked by $Z^{(i)}, i\in [0, k]$. We learn them as graph embeddings of the entities by solving Eq~\ref{eq:LE}, which is encoded from the encoder $\phi_\theta^T(X_{\text{\textit{data}}})$ with parameters $\theta$. 

\begin{equation}\label{eq:LE}
\begin{aligned}
   & \min_{\mathcal{Z}: \mathcal{Z}^TD\mathcal{Z}=I} Tr(\mathcal{Z}^T(D- w)\mathcal{Z}) \\
   = & \min_{\mathcal{\theta}: \phi_\theta^T(X_{\text{\textit{data}}})D\phi_\theta(X_{\text{\textit{data}}})=I} Tr(\phi_\theta^T(X_{\text{\textit{data}}})(D- w)\phi_\theta(X_{\text{\textit{data}}})) \\
   & \text{where} \ \ \ \ \  D = \text{diag}(\sum_j w_{ij})
\end{aligned}
\end{equation}

This can be solved as a generalized eigenvalue problem~\cite{belkin2003laplacian}. The closed-formed solution is $k$ eigenvectors of the Laplacian matrix $L = D -w$ with minimal eigenvalues that are not zero. To generalize the discussion to unseen data, we further discuss how to generalize the problem to extrapolate to infinite new samples sampled according to the data distribution.
% , how aligning representations of different transform-encoder compositions affects the connectivity of the nodes in graph $\mathcal{G}$, and the designation of corresponding training objective in \textbf{Sec.}~\ref{sec:algo}.%}.\footnote{where to discuss this problem and what's their relationship?}

Finally, only the shapelet neural encoder~\cite{liang2023contrastive} for the raw MTS data is preserved during adaptations for downstream tasks and inferring processes. This raw data feature encoder memorizes algorithmic knowledge from the transforms and encoders. The scalability of the method is also promoted by the small scale of the neural architecture during inference. 




%%开始应当说这节解决的问题,和前面的关系,基本思想以及roadmap,然后再按照roadmap的顺序依次介绍本节内容
%%这章重新整理一下吧


% \section{transform Operators and Feature Extractors Selection}\label{sec:tran}

% In this section, our goal is to leverage previously introduced transform operators $T^{(i)}, i\in[k]$, to derive multi-modal feature entities capable of revealing salient discriminative patterns. Concurrently, we select neural encoders $f^{(i)}, i\in [k]$, adept at capturing these inherent patterns within the transformed features. This enables the encoders to approximate the eigenfunctions of the global alignment graph, and project input data into a lower-dimensional space which is conducive to natural clustering based on pattern occurrence and graph connectivity.

% Hence, the process of selecting transform operators $T^{(i)}$ and their corresponding neural encoders, necessitates assumptions about the discriminative patterns present in the datasets under examination. These assumptions about inductive bias fundamentally address the question of \textit{how an algorithm prioritizes learning one pattern over another.} Consequently, the pivotal principle arises: for datasets exhibiting diverse characteristics and varying sizes, the compatibility of pairings becomes pivotal in providing effective supervisory signals. To streamline, we focus on several potent compositions, detailed in \textbf{Tab.}~\ref{tab:TFpairs}.

% As discussed in \textbf{Sec.}\ref{sec:rela}, time series data pose complexities in pattern identification and analysis, characterized by heterogeneous feature entities and their diverse patterns. For instance, IMU signals exhibit multiple periodic swings with rapidly varying patterns across different temporal stages\cite{bagnall2018uea}. These traits collectively render the signals informative yet challenging pattern extraction. Notably, the DFT directly extracts frequency location patterns in the spectral domain while disregarding temporal stages, emphasizing contributions for certain frequencies across the time axis. Conversely, biological electric signals manifest a blend of frequencies challenging to discern without temporal domain transforms. Specific visual examples illustrating temporal stage changes and frequency mixing are demonstrated in \textbf{Fig.}\ref{fig:contrast}.

% The selection of feature encoders is generally contingent upon the chosen transform operator. For instance, we employ CNNs as encoders for FFT and transforms that convert each channel of time series data into a two-dimensional matrix. Longformer models are employed to encode symbolic features. Additionally, a neural feature encoder can be trained from scratch or pre-trained with external knowledge injected, as expounded in \textbf{Sec.}~\ref{sec:empi}.

% Thus, nine compositions of transform operators and feature encoders, outlined in \textbf{Tab.}~\ref{tab:TFpairs}, are designed to hypothesize discriminative features within datasets, introducing varied inductive biases to the primary encoder. In the subsequent subsection, we demonstrate approaches to align representations extracted from each composition and elucidate two mechanisms for MMFA, introducing inductive bias for enhanced representation learning.

% \begin{figure}
%     \centering
%     \includegraphics[width=0.25\textwidth]{graph.pdf}
%     \caption{\textbf{Global alignment edges between sub-graphs preserving local arrangement, changing connectivity of the whole graph.} Dashed lines denote nodes accessible via local fluctuation. The solid lines indicate edges between nodes transformed by the same samples, which globally align features with different modalities.}
%     \label{fig:graph}
% \end{figure}

\section{Regularization Based Multi-Modal Feature Alignment URL Training Algorithm}\label{sec:algo}

In this section, we aim to build embeddings that can capture all of the inductive biases introduced, which are learned from a global alignment graph $\mathcal{G}$. 

Initially, we analyze the approximated eigenfunctions of the graph Laplacian operator, each of which partitions the representation space into different clusters. Subsequently, we introduce training goals to acquiring representations of the input data from these approximated eigenfunctions. The supervisory goal can be seen as learning \textit{graph Laplacian embedding (GLE)}~\cite{trillos2021geometric}. We then design an algorithm to minimize the training objective. Representations encoded from multi-modal features transformed from raw time series are finally well arranged.

%%这节的roadmap


\subsection{Training Objective Approximating GLE}\label{sec:desi}

%%先说这个section解决的问题,再说基本思想
This section aims to formulate the URL problem in a manner akin to learning graph Laplacian embedding (GLE) on the multi-modal alignment graph, which gives global information about the data manifold. Since we treat the neural encoders as continuous mappings, they automatedly arrange representations according to local fluctuations. As is shown in \textbf{Fig.}~\ref{fig:contrast}, different transforms provide distinct local properties that would saliently influence the local arrangement. Initially, we establish fundamental concepts closely associated with the GLE problem. Subsequently, we introduce the training objectives.

Training and testing can be seen as independently and uniformly sample data samples with a distribution $p_{\text{\textit{data}}}^\prime$ as entities of interest. The training procedure can be seen as sampling jointly from a probability distribution which can approximate and match the equivalence of pairs of the entities, $p_{\text{\textit{sim}}}$, whose possibility density function (PDF) can be constructed as a piecewise function, $ \pi = \frac 1 k p_{T_{(x_i)}, T_{(x_j)}}(x_i, x_j): \mathcal{S}^{\text{\textit{data}}} \times \mathcal{S}^{\text{\textit{data}}} \to [0,1]$, where $T_{(x_m)}=T^{(n)}$ denotes the transform used to acquire $x_m \in X^{(n)}$, and $\mathcal{S}^{\text{\textit{data}}} = \sqcup_{i \in [0,k]} \text{Im}(T^{(i)})$ which we fist introduce in \textbf{Sec.}~\ref{sec:neur_encoder_main}. Since the different transform pairs are sampled equally, the joint distributions for them are simply sliced and normalized. Next, we can generalize $\mathcal{G}$ into a graph describing any of the entities conforms $p_{\text{\textit{data}}}^\prime: \mathcal{S}^{\text{\textit{data}}} \to \mathbb{R}$, and weighted edges between them. Since the GLE problem is strongly related to the concept of the Laplacian operator and its eigenfunction, we next define them in Def.~\ref{def:laplace} and Def.~\ref{def:lap_eig}, respectively.

%%太突兀,引出这个定义,说明定义他的动机

\begin{definition}\label{def:laplace}
For a function $f: \mathcal{S}^{\text{\textit{data}}} \to \mathbb{R}$, the \emph{Laplacian operator} $\mathbb{L}$ for PDF $\pi$ is defined as:

\begin{equation}\label{eq:laplacian}
    \begin{aligned}
\mathbb{L}(f)(x)  = f(x) + \int_{x^\prime} \frac {\frac 1 k p_{T_{(x)}, T_{(x^\prime)}}(x, x^\prime)} {p_{\text{\textit{data}}}^\prime(x)} f(x^\prime) dx^\prime \\
  \  = f(x) + \frac 1 k \sum_{i=1}^k \int_{x^\prime \in T^{(i)}(\overline{\mathcal{X}})} \frac {p_{T_{(x)}, T_i}(x, x^\prime)} {p_{\text{\textit{data}}}^\prime(x)} f(x^\prime) dx^\prime    
    \end{aligned}
\end{equation}
\end{definition}

\begin{definition}\label{def:lap_eig}
An \emph{eigenfunction} of the Laplacian operator $\mathbb{L}$ with eigenvalue $\lambda$ is a function $f$ satisfies Eq.~\ref{eq:eigen}:

\begin{equation}\label{eq:eigen}
    \mathbb{E}_{x\sim p_{\text{\textit{data}}}^\prime}[(\mathbb{L}(f)(x) - \lambda f(x))^2]=0 
\end{equation}
\end{definition}

Generally, eigenfunctions with small eigenvalues would correspond to clusters that are almost disconnected from the rest of the graph, which is well-known in the spectral graph theory~\cite{trevisan2017lecture}. 

Based on this and the Laplacian operator $\mathbb{L}$ definition, MMFA enhances base encoder representations. The chosen transform-encoder compositions construct sub-graphs from raw MTS, each revealing distinct discriminative patterns. The enhancement occurs through two mechanisms.

Firstly, the transform-encoder compositions show the robustness of irrelevant patterns that are introduced by each other. The alignment method we proposed assigns high priority to learning these transform properties. Secondly, some of the important patterns may be ignored by certain compositions but can be enhanced by another one. The added edges help to inject deduced invariance from one composition to the others. These added edges are introduced by $\pi$. \textbf{Theorem}~\ref{the:seman_sim} models more connected components of the graph to eigenvectors by reducing the distance between representations of entities at each end of the edges.
%%引出定理,要有解释

\begin{theorem}\label{the:seman_sim}
Equivalence between Eigenvalues and Distance Reduction of Spectral Embeddings.

$\mathbb{E}_{(x, x^\prime)\sim p_{\text{\textit{sim}}}} [(f(x) - f(x^\prime))^2]$ denotes the expected squared difference between representations of data point pairs under distribution $p_{\text{\textit{sim}}}$, and $f$ is an eigenfunction of $\mathbb{L}$. Eq.~\ref{eq:distance} demonstrates the relationship as follows:

\begin{equation}\label{eq:distance}
    \mathbb{E}_{(x, x^\prime)\sim p_{\text{\textit{sim}}}} [(f(x) - f(x^\prime))^2] = 2\lambda\mathbb{E}_{x \sim p_{\text{\textit{data}}}^\prime}[f(x)^2]
\end{equation}

Proof of the theorem is in \textbf{Sec.}~\ref{proof:eq}.
\end{theorem}



\begin{remark}
\textbf{Theorem}~\ref{the:seman_sim} illustrates that a lower eigenvalue $\lambda$ correlates with reduced distance between nearly aligned pairs. Consequently, minimizing the distance during training aligns the encoder with the eigenfunctions of the Laplacian operator possessing small eigenvalues, effectively segregating the graph-captured relations into relatively distinct clusters.
\end{remark}

We denote raw MTS samples and their distribution as $x \sim p_{\text{\textit{data}}}$. $X$ is a set of samples sampled with $p_{\text{\textit{data}}}$. To capture the global invariance implied in $w$, the corresponding loss can be estimated and bounded by Inequation~\ref{eq:invprime}, where $Z \in \mathbb{R}^{(k+1)\times N \times d }$ which is reshaped from $\mathcal{Z}$, and $z_i^{(m)} \in Z^{(m)}$ denotes the representation for the $i$th sample transformed and encoded by $T^{(m)}$ and $f^{(m)}$. 

To ensure the feasibility of training, we provide an upper bound of the originally proposed loss function to reduce the training workload, which is demonstrated at \textbf{Theorem}~\ref{the:inv_est}.
%%引出下面的定理

\begin{theorem}\label{the:inv_est}
Multi-modal Invariance Estimation.

$x \sim p_{\text{\textit{data}}}$ denotes raw Multivariate Time Series (MTS) samples, where $X$ represents a group of samples sampled with $p_{\text{\textit{data}}}$. Strictly, $p_{T^{(i)}, T^{(j)}}(T^{(i)}(x), T^{(j)}(x)) = 1, x \in X$. This captures the global equivariance implied in $w$, the corresponding loss, denoted by $\mathcal{L}_{\text{inv}}^\prime(Z)$, can be estimated and bounded as shown in Inequality $\ref{eq:invprime}$.

\begin{equation}\label{eq:invprime}
\begin{aligned}
    \mathcal{L}_{\text{inv}}^\prime(Z) & = \frac{1}{Nk(k+1)}\sum_{\substack{m=0 \\ n=0 \\ m \neq n}}^{k} \sum_{i=1}^N ||z_i^{(m)} - z_i^{(n)}||^2_2 \\
    & \leq \frac{2}{N(k+1)} \sum_{i=1}^N\sum_{m=1}^k ||z^{(0)}_i - z^{(m)}_i||^2_2
\end{aligned}
\end{equation}

Proof of the theorem is in \textbf{Sec.}~\ref{proof:inv}.
\end{theorem}
The expression \( \mathcal{L}_{\text{inv}}^\prime(Z) \) simplifies the optimization objective aimed at capturing global equivariance among the multi-modal features.


According to the \textbf{Theorem}~\ref{the:inv_est}, we define $\mathcal{L}_{inv}(Z)$ as the optimization objective capturing global invariance among the multi-modal features.

\begin{equation}\label{eq:inv}
    \mathcal{L}_{inv}(Z) = \frac{1}{N(k+1)} \sum_{i=1}^N \sum_{m=1}^k ||z^{(0)}_i - z^{(m)}_i||^2_2
\end{equation}

\begin{theorem}\label{the:orth}
Orthogonality of Eigenfunction-Based Representations.

For $ \forall g, h: \mathbb{R^{\text{\textit{data}}}} \to \mathbb{R}$, $\mathbb{E}{x \sim p_{\text{\textit{data}}}'}[\mathbb{L}(g)(x) \cdot h(x)] =\mathbb{E}{x \sim p_{\text{\textit{data}}}'}[g(x) \cdot \mathbb{L}(h)(x)]$. Therefore, $\mathbb{L}$ is a symmetric linear operator. The representations are outputs derived from approximated eigenfunctions of $\mathbb{L}$ possessing low eigenvalues, $f_{\text{rep}}(\cdot) = [f_1(\cdot), f_2(\cdot), \dots, f_{d_z}(\cdot)]^T$, and $\mathbb{L}$ is a symmetric real value linear operator. According to spectral theorem~\cite{trevisan2017lecture}, it is feasible to ensure orthogonality while maximizing the overall information:


\begin{algorithm}[t]
   \caption{Asymmetric Neural Encoders Alignment}
   \label{algo}
   \begin{algorithmic}
   \STATE {\bfseries Input:} $D$: MTS dataset; $T^{(i)}, i \in [k]$: transform operators; $f^{(i)}, i \in [k]$: neural encoder architectures.
   \STATE {\bfseries Output:} $f$: Trained ST encoder for MTS representation.
   
   \STATE $\theta_{f} \gets$ Initialize parameters of the raw MTS encoder.
   \FOR{$i \in [k]$}
       \STATE $\theta_{f^{(i)}} \gets$ Initialize parameters of the encoders w.r.t $T^{(i)}$.
   \ENDFOR
   \WHILE{not converged}
       \STATE $X \gets$ Sample a mini-batch from $D$.
       \FOR{$i \in [k]$}
           \STATE $X^{(i)} \gets T^{(i)}(X)$ \# Transform raw MTS.
           \STATE $\theta_{\tilde f}, \theta_{\tilde f^{(i)}} \gets U(f, f^{(i)}, X, X^{(i)})$ \# Accumulate gradients.
           \STATE $\theta_{f^{(i)}} \gets \theta_{f^{(i)}} + \epsilon \frac{1}{k} (\theta_{\tilde f^{(i)}} - \theta_{f^{(i)}})$ \# Update encoder.
       \ENDFOR
       \STATE $\theta_{f} \gets \theta_{f} + \epsilon (\theta_{\tilde f} - \theta_{f})$ \# Update encoder.
   \ENDWHILE
   \STATE {\bfseries Return:} $f$
   \end{algorithmic}
\end{algorithm}

\begin{equation}
\mathbb{E}_{x \sim p_{\text{\textit{data}}}'}[f_{\text{rep}}(x)f_{\text{rep}}(x)^T] = \mathbb{I}
\end{equation}
\end{theorem}

According to \textbf{Theorem}~\ref{the:orth}, we design two relevant losses, $\mathcal{L}_{var}$ and $\mathcal{L}_{cov}$, in Eq.~\ref{eq:var} and Eq.~\ref{eq:cov}, respectively.

\begin{equation}\label{eq:var}
\begin{aligned}
    & \mathcal{L}_{cov}(Z) \\
    & = \frac 1 {(N - 1)kd} \sum_{l=1}^k \sum_{m \neq n} [\sum_{i=1}^N(z_{i}^{(l)} - \bar z^{(l)})(z_{i}^{(l)} - \bar z^{(l)})^T]_{m,n} \\ 
    & where \ \bar z^{(l)} = \frac 1 n \sum_{i=1}^n z_{i}^{(l)}
\end{aligned}
\end{equation}

\begin{equation}\label{eq:cov}
    \mathcal{L}_{var}(Z) = \frac 1 {dk} \sum_{l=0}^k \sum_{j=1}^d max(0, 1 - \sqrt{Var(\{z^{(l)}_{i,j} | i\in [N]\}) + \epsilon})
\end{equation}
According to Eq.~\ref{eq:inv}, Eq.~\ref{eq:var} and Eq.~\ref{eq:cov}, the end-to-end training object is obtained in equation~\ref{eq:loss}. $\alpha$, $\beta$ and $\gamma$ are empirical parameters.

\begin{equation}\label{eq:loss}
    \mathcal{L} = \alpha \mathcal{L}_{cov} + \beta \mathcal{L}_{var} + \gamma \mathcal{L}_{inv}
\end{equation}


Our further discussion deduced the closed-form solutions of linear degenerated MMFA with $k=0$ in \textbf{Lemma}~\ref{lem:PCA}, $k=1$ in \textbf{Lemma}~\ref{lem:CCA}, and $k=p$ in \textbf{Lemma}~\ref{lem:multiCCA} transforms aligned with the identity transformation in \textbf{Sec.}~\ref{sec:recover}.  According to the closed-form optimal solutions, we can observe that MMFA serves as a spectral selection mechanism for balancing variances of empirical distributions measured with the transformed features.

\begin{theorem}\label{theorem:recover}
    The optimization problem of MMFA recovers kernel principal component analysis (KPCA)~\cite{kpca} and kernel canonical correlation analysis (KCCA)~\cite{fukumizu2007statistical}.

    Proof of the theorem is in \textbf{Sec.}~\ref{sec:recoverk}.
\end{theorem}


\begin{table*}[t]
    
    \caption{Comparisons of performance in MTS classification, evaluated on accuracy scores. The top-performing results of URL methods are shown in bold, with $\dag$ denoting the overall best performance.}
    \small
    \label{tab:classification}
    \resizebox{1\textwidth}{!}{
    \begin{tabular}{lcccccc|cccccccc}
    \toprule
    \multirow{2}{*}{\textbf{Dataset}} & \multicolumn{6}{c|}{\textbf{Tailored Classification Approaches}} & \multicolumn{8}{c}{\textbf{Unsupervised Representation Learning + Classifier}} \\
    \cline{2-15}

    & \method{DTWD}{bagnall2018uea} 
& \method{MLSTM-FCNs}{karim2019multivariate} 
& \method{TapNet}{zhang2020tapnet} 
& \method{ShapeNet}{li2021shapenet} 
& \method{OSCNN}{tang2020omni} 
& \method{DSN}{xiao2022dynamic} 
& \method{TS2Vec}{yue2022ts2vec} 
& \method{T-Loss}{franceschi2019unsupervised} 
& \method{TNC}{tonekaboni2021unsupervised} 
& \method{TS-TCC}{eldele2021time} 
& \method{TST}{zerveas2021transformer} 
& \method{CSL}{liang2023contrastive}
 &  MMFA-aug & \textbf{MMFA}  \\
    \midrule
    	
ArticularyWordRecognition   & 0.987   & 0.973        & 0.987    & 0.987         & 0.988        & 0.984         & 0.987                 & 0.943          & 0.973        & 0.953            & 0.977  & 0.990    &  0.973 & \textbf{0.993}$^\dag$\\
AtrialFibrillation   & 0.200     & 0.267        & 0.333    & 0.400	        & 0.233	       & 0.067         & 0.200	               & 0.133	        & 0.133	       & 0.267	          & 0.067	  &  0.533    & 0.533 &  \textbf{0.600}$^\dag$\\
BasicMotions   & 0.975	  & 0.950	    & 1.000$^\dag$ & 1.000$^\dag$      & 1.000$^\dag$     & 1.000$^\dag$      & 0.975	             &\textbf{1.000}$^\dag$	& 0.975	       & \textbf{1.000}$^\dag$& 0.975	& \textbf{1.000}$^\dag$ & \textbf{1.000}$^\dag$      & \textbf{1.000}$^\dag$ \\
CharacterTrajectories  & 0.989	  & 0.985	     & 0.997	& 0.980	        & 0.998$^\dag$ & 0.994         & \textbf{0.995}        & 0.993	        & 0.967	       & 0.985	          & 0.975	 & 0.991    & 0.987     & 0.991 \\
Cricket   &1.000$^\dag$ & 0.917      	 & 0.958	& 0.986	        & 0.993	       & 0.989         & 0.972	               & 0.972	        & 0.958	       & 0.917	      &\textbf{1.000}$^\dag$  & 0.994  & 0.986  & \textbf{1.000}$^\dag$  \\
DuckDuckGeese   & 0.600     & 0.675	     & 0.575	& 0.725$^\dag$  & 0.540	       & 0.568         & \textbf{0.680}         & 0.650	        & 0.460	       & 0.380	          & 0.620	& 0.380    & 0.620   & \textbf{0.680}            \\
EigenWorms  & 0.618	  & 0.504	   	 & 0.489	& 0.878$^\dag$	& 0.414        & 0.391         & \textbf{0.847}        & 0.840	        & 0.840	       & 0.779	          & 0.748	& 0.779   & 0.768    & 0.840           \\
Epilepsy   & 0.964	  & 0.761	    	 & 0.971	& 0.987	        & 0.980	       & 0.999$^\dag$  & 0.964	        & 0.971	        & 0.957	       & 0.957	          & 0.949  & 0.986	 & 0.899     & \textbf{0.987}$^\dag$ \\
ERing   & 0.133	  & 0.133	    	 & 0.133	& 0.133	        & 0.882	       & 0.922         & 0.874	               & 0.133	        & 0.852	       & 0.904	          & 0.874	 & \textbf{0.967}$^\dag$   &  0.922  & \textbf{0.967}$^\dag$  \\
EthanolConcentration  & 0.323	  & 0.373	   	 & 0.323	& 0.312	        & 0.241	       & 0.245         & 0.308	               & 0.205	        & 0.297	       & 0.285	          & 0.262	 & 0.498    &       0.331  &  \textbf{0.551}$^\dag$ \\
FaceDetection  & 0.529	  & 0.545	    	 & 0.556	& 0.602	        & 0.575	       & 0.635$^\dag$  & 0.501	               & 0.513	        & 0.536	       & 0.544	          & 0.534	& \textbf{0.593}  & 0.526    & 0.555        \\
FingerMovements  & 0.530	  & 0.580	    	 & 0.530	    & 0.580	        & 0.568	       & 0.492         & 0.480	               & 0.580	        & 0.470	       & 0.460	          & 0.560	 & 0.590  & 0.490    & \textbf{0.610}$^\dag$  \\
HandMovementDirection  & 0.231	  & 0.365	   	 & 0.378	& 0.338	        & 0.443 & 0.373         & 0.338	               & 0.351	        & 0.324	     & 0.243	          & 0.243    &    0.432   &  0.351 &
 \textbf{0.487}$^\dag$\\
Handwriting  & 0.286	  & 0.286	    	 & 0.357	& 0.451	        & 0.668$^\dag$  & 0.337         & 0.515	               & 0.451	        & 0.249	       & 0.498	          & 0.225	& \textbf{0.533}   & 0.370     & 0.472    \\
Heartbeat  & 0.717	  & 0.663	    	 & 0.751	& 0.756	        & 0.489	       & 0.783$^\dag$  & 0.683	               & 0.741	        & 0.746	       & 0.751	  & 0.746	& 0.722      &  0.720 &   \textbf{0.761} \\

InsectWingbeat  & N/A	      & 0.167	   	 & 0.208	& 0.250	        & 0.667$^\dag$ & 0.386         & 0.466                 & 0.156	        &\textbf{0.469} & 0.264	          & 0.105	 & 0.256      & 0.425    & \textbf{0.469}       \\  

JapaneseVowels  & 0.949	  & 0.976	    	 & 0.965	& 0.984	        & 0.991$^\dag$ & 0.987         & 0.984	               & \textbf{0.989} & 0.978	       & 0.930	          & 0.978 & 0.919	 &  0.960  & 0.978     \\
Libras & 0.870	  & 0.856	   	 & 0.850	    & 0.856	        & 0.950	       & 0.964$^\dag$  & 0.867	               & 0.883	        & 0.817	       & 0.822	          & 0.656	& \textbf{0.906} & 0.833    & 0.894   \\
LSST  & 0.551	  & 0.373	    	 & 0.568    & 0.590	        & 0.413	       & 0.603         & 0.537	               & 0.509	        & 0.595	       & 0.474	          & 0.408	& 0.617    & 0.372     & \textbf{0.622}$^\dag$    \\
MotorImagery & 0.500	  & 0.510	    	 & 0.590	    &0.610$^\dag$    & 0.535	       & 0.574         & 0.510	               & 0.580	        & 0.500	       & 0.610  & 0.500	  &  0.610 & 0.510   & \textbf{0.630}$^\dag$  \\
NATOPS  & 0.883	  & 0.889	    	 & 0.939	& 0.883	        & 0.968	       & 0.978$^\dag$        & 0.928        & 0.917          & 0.911	       & 0.822	          & 0.850	& 0.878     & 0.900     & \textbf{0.933}         \\
PEMS-SF  & 0.711	  & 0.699	   	 & 0.751	& 0.751	        & 0.760	       & 0.801         & 0.682	               & 0.676	        & 0.699	       & 0.734	          & 0.740	& 0.827      & 0.827      & \textbf{0.838}$^\dag$ \\

PenDigits  & 0.977	  & 0.978	     & 0.980	    & 0.977	        & 0.986	       & 0.987         & 0.989	               & 0.981	        & 0.979	       & 0.974	          & 0.560 & \textbf{0.990}$^\dag$	& 0.979   & 0.980    \\
PhonemeSpectra  & 0.151	  & 0.110	    	 & 0.175	& 0.298	        & 0.299$^\dag$ & 0.320          & 0.233	               & 0.222	        & 0.207	       & 0.252	          & 0.085	& \textbf{0.255} & 0.183     & 0.216  \\
RacketSports  & 0.803	  & 0.803	    	 & 0.868	& 0.882$^\dag$	& 0.877	       & 0.862         & 0.855	               & 0.855	        & 0.776	       & 0.816	          & 0.809  & \textbf{0.882}$^\dag$  & 0.815   & 0.862 \\
SelfRegulationSCP1  & 0.775	  &0.874		 & 0.652	& 0.782	        & 0.835	       & 0.717         & 0.812	               & 0.843	        & 0.799	       & 0.823	          & 0.754 &  0.846 &  0.887	    &  \textbf{0.901}$^\dag$  \\
SelfRegulationSCP2  & 0.539	  & 0.472	    	 & 0.550	    & 0.578     	& 0.532	       & 0.464         & 0.578            & 0.539	        & 0.550	       & 0.533	          & 0.550 & 0.496	 & 0.617     &  \textbf{0.628}$^\dag$  \\
SpokenArabicDigits  & 0.963	  & 0.990	    	 & 0.983	& 0.975	        & 0.997$^\dag$ & 0.991         & 0.988	               & 0.905	        & 0.934	       & 0.970	          & 0.923	& \textbf{0.990}    &  0.980   & \textbf{0.990} \\
StandWalkJump  & 0.200	  & 0.067      	 & 0.400	    & 0.533	        & 0.383	       & 0.387         & 0.467	               & 0.333	        & 0.400	       & 0.333	          & 0.267	     & \textbf{0.667}$^\dag$  & \textbf{0.667}$^\dag$  &  \textbf{0.667}$^\dag$ \\
UWaveGestureLibrary  & 0.903	  & 0.891	      & 0.894	& 0.906	        & 0.927 & 0.916         & 0.906	               & 0.875	        & 0.759	       & 0.753	          & 0.575	& 0.922    & 0.922  &  \textbf{0.931}$^\dag$ \\
 
\hline
\makecell[l]{\textbf{Avg Cohen's d Effect}}  \\
incl. InsectWingbeat & /	  & -0.670	    	 & -0.024	& 0.366	        & 0.357	       & 0.280         & 0.089  & -0.176 & -0.450 & -0.447 & -0.860  & 0.502 & 0.096 & 
\textbf{0.937}$^\dag$  \\

excl. InsectWingbeat & -0.376	  & -0.631	   & 0.033	& 0.434	        & 0.321	       & 0.308         & 0.094  & -0.116 &  -0.473 & -0.425 & -0.825  & 0.567 & 0.109 & \textbf{0.979}$^\dag$  \\
\hline
%\textbf{Avg Ranking} & & &  & & & & & & & & & \\
\makecell[l]{\textbf{Avg Ranking} (URL only)} & / & / & / & / & / & / & 3.96 & 4.63 & 5.40  & 5.30 &  6.20 & 2.83 & 4.43 & \textbf{1.60} \\
\makecell[l]{\textbf{Avg Ranking} (All)} & 
9.40 & 9.43 & 6.90 &  5.43 & 5.37 &
   5.83 & 6.83 & 7.90 & 9.37  & 9.10 &
       10.46 & 4.40 & 7.17  & \textbf{2.63}$^\dag$ \\
    \bottomrule
    \end{tabular}
    }
\end{table*}



\subsection{Asymmetric Encoders Alignment Optimizing Algorithm}

We finally design Algorithm~\ref{algo} minimizing the training objective by mini-batch gradient descent. Algorithm~\ref{algo} takes the MTS dataset $D$, a set of transform operators $T^{(i)}$, and corresponding neural architecture for the encoders as input. Then the parameters of the encoders $f$ and $f^{(i)}$ are initialized in lines 1-3. In lines 7-9, update step $U$ is defined as $U(f, f^{(i)}, X, X^{(i)}) = \theta - \nabla_\theta \mathcal{L}([f^{(i)}(X^{(i)}), f(X)])$, where $\theta$ denotes the parameters of the two encoders. We divide the designed training objective into $k$ parts, iteratively computes and accumulates gradient for both $f$ and $f^{(i)}$. After accumulating all the gradients of the batch, the encoder $f$ is updated.

% We employ a multi-stage learning approach to align the representations learned by individual encoders with the main time series encoder, thereby facilitating the creation of a cohesive and informative representation of MTS data suitable for downstream tasks.

After the training phase, we only need to preserve the encoder, $f$, for raw MTS and adapt it to diverse downstream tasks.



\section{Empirical Analysis}\label{sec:empi}

In this section, we conduct extensive experiments containing 35 real-world datasets for three important downstream tasks, classification, clustering, and anomaly detection. The 30 datasets for classification and 12 datasets for clustering vary with their \textit{training set sizes}, \textit{dimensionalities}, \textit{lengths}, and \textit{numbers of classes} which are detailed in \textbf{Tab.}~\ref{tab:uea_statistics}. The remaining 5 datasets for anomaly detection have different \textit{numbers of entities}, \textit{dimensionality}, \textit{length}, and \textit{anomaly ratio} which are detailed in \textbf{Tab.}~\ref{tab:ad_statistics}. Standard metrics are used for various tasks, i.e., Accuracy (Acc)~\cite{bagnall2018uea} for classification, Cohen's d~\cite{becker2000effect} for standardizing performance on various datasets, RI~\cite{Hubert1985ComparingP} and NMI~\cite{Strehl2002ClusterE} for clustering, and F1-score~\cite{li2021multivariate} for anomaly detection.

The baselines are divided into two groups,i.e., the URL methods and methods tailored to specific tasks. All the URL methods share the same evaluation protocol on downstream tasks, by feeding the representations to classical models, i.e., SVM~\cite{Cortes1995SupportVectorN}, IF~\cite{Liu2008IsolationF}, and KMeans~\cite{MacQueen1967SomeMF}. The tailored methods are evaluated following the original implementations. More detailed demonstrations of the downstream tasks, datasets, evaluation metrics, and baselines are elaborated in \textbf{Sec.}~\ref{sec:exp_detail}.




% \subsection{Empirical Analysis}\label{sec:empi}

We evaluate the robustness and adaptability of MMFA across diverse datasets and domains. Through experiments, we aim to scrutinize algorithms, assess scalability, and compare them to the proposed MMFA framework, outlined by our research questions (RQs). Two important questions are shown below and the others are discussed in \textbf{Sec.}~\ref{sec:additional_RQs}.



\begin{table*}[h]
    \centering
    \caption{The MTS clustering performance evaluated on RI~\cite{Hubert1985ComparingP} and NMI~\cite{Strehl2002ClusterE}, with the most effective results from URL methods in bold, with $\dag$ indicating superiority over others.}
    \label{tab:clustering}
    \resizebox{1\linewidth}{!}{
    \begin{tabular}{lcccccccc|cccccccc}
    \toprule
    \multirow{2}{*}{\textbf{Dataset}} & \multirow{2}{*}{\textbf{Metric}} & \multicolumn{7}{c|}{\textbf{Tailored Clustering Approaches}} & \multicolumn{6}{c}{\textbf{Unsupervised Representation Learning + Clustering}} \\
    \cline{3-17}&
      & \method{MC2PCA}{li2019multivariate} & \method{TCK}{mikalsen2018time} & \method{m-kAVG+ED}{ozer2020discovering} & \method{m-kDBA}{ozer2020discovering} & \method{DeTSEC}{ienco2020deep} & \method{MUSLA}{zhang2022multiview} & \method{Time2Feat}{bonifati2022time2feat} & \method{TS2Vec}{yue2022ts2vec} & \method{T-Loss}{franceschi2019unsupervised} & \method{TNC}{tonekaboni2021unsupervised} & \method{TS-TCC}{eldele2021time} & \method{TST}{zerveas2021transformer} & \method{CSL}{liang2023contrastive} & MMFA-aug & \textbf{MMFA} \\
    \midrule
\multirow{2}{*}{ArticularyWordRecognition} & RI	& 0.989 & 0.973 &	0.952 &	0.934 &	0.972  &	0.977 &   0.977 &	0.980 &	0.975 &	0.938 &	0.946 &	0.978 & 0.990 &	0.990  & \textbf{0.998}$^\dag$     \\
                                           & NMI & 0.934 &	0.873 &	0.834 &	0.741 &	0.792  &	0.838 & 0.881 &	0.880 &	0.842 &	0.565 &	0.621 &	0.866 &	0.942 & 0.942& \textbf{0.983}$^\dag$       \\
\multirow{2}{*}{AtrialFibrillation} & RI	& 0.514 & 0.552 &  	0.705 &	0.686 &	0.629  &	0.724 & 0.600 &  0.465 &	0.469 &	0.518 &	0.469 &	0.444 &	\textbf{0.743}$^\dag$  &	0.524  & 0.648  \\
                                           & NMI & 0.514	& 0.191 &  	0.516 &	0.317 &	0.293  &	0.538 &	0.261 & 0.080 &	0.149 &	0.147 &	0.164 &	0.171 &	\textbf{0.587}$^\dag$& 0.251 & 0.305    \\
\multirow{2}{*}{BasicMotions} & RI & 0.791	& 0.868 &  	0.772 &	0.749 &	0.717  &	1.000$^\dag$ & 0.883 &	0.854 &	0.936 &	0.719 &	0.856 &	0.844 & \textbf{1.000} &	0.845    &  \textbf{1.000}      \\
                                           & NMI & 0.674	& 0.776 &  	0.543 &	0.639 &	0.800  &	1.000$^\dag$ & 0.833 &	0.820 &	0.871 &	0.394 &	0.823 &	0.810 & \textbf{1.000} & 0.806 & \textbf{1.000}    \\
\multirow{2}{*}{Epilepsy} & RI & 0.613	& 0.786 &  	0.768 &	0.777 &	0.840  & 0.816 &	0.897$^\dag$ & 0.706 &	0.705 &	0.650 &	0.736 &	0.718 & 0.873  &	0.679    & \textbf{0.876}   \\
                                           & NMI & 0.173	& 0.534 &  	0.409 &	0.471 &	0.346  &	0.601 &	0.814$^\dag$ & 0.312 &	0.306 &	0.156 &	0.451 &	0.357 & 0.705  & 0.530  & \textbf{0.721}    \\
\multirow{2}{*}{ERing} & RI	& 0.756 & 0.772 &  	0.805 &	0.775 &	0.770  &	0.887  & 0.841  & 0.925 &	0.885 &	0.764 &	0.821 &	0.867 & 0.968  &	0.968  & \textbf{0.972}$^\dag$  \\
                                           & NMI & 0.336	& 0.399 &  	0.400 &	0.406 &	0.392  &	0.722 &	0.641 &0.775 &	0.672 &	0.346 &	0.478 &	0.594 &	0.906 & 0.906 & \textbf{0.918}$^\dag$   \\
\multirow{2}{*}{HandMovementDirection} & RI & 0.627	& 0.635 &  	0.697 & 0.685 &	0.628  &	0.719$^\dag$ & 0.577 & 0.609 	& 0.599 & 0.613 &	0.608 &	0.607 & 0.651 &	0.630 & \textbf{0.657} \\
                                           & NMI & 0.067	& 0.103 &  	0.168 &	0.265 &	0.112  &	0.398$^\dag$ & 0.062 &	0.044 &	0.034 &	0.051 &	0.053 &	0.039 &	0.175 & 0.118  &	\textbf{0.185} \\
\multirow{2}{*}{Libras} & RI & 0.892 	& 0.917 &  	0.911 &	0.913 &	0.907  &	0.941 &	0.926 & 0.904 &	0.922 &	0.896 &	0.881 &	0.886 & 0.941 &	0.938 & \textbf{0.944}$^\dag$ \\
                                           & NMI	& 0.577 & 0.620 &  	0.622 &	0.622 &	0.602  &	0.724 &	0.703 & 0.542 & 0.654 &	0.464 &	0.373 &	0.400 &	0.761 & 0.730 & \textbf{0.785}$^\dag$    \\
\multirow{2}{*}{NATOPS} & RI & 0.882	& 0.833 &  	0.853 &	0.876 &	0.714  &	0.976$^\dag$ & 0.777 &	0.817 &	0.836 &	0.700 &	0.792 &	0.809 &	0.876 &	0.850  &  \textbf{0.919}    \\
                                           & NMI & 0.698	& 0.679 &  	0.643 &	0.643 &	0.043  &	0.855$^\dag$ & 0.482 &	0.523 &	0.558 &	0.222 &	0.513 &	0.565 &	0.657 &	0.628   & \textbf{0.864} \\
\multirow{2}{*}{PEMS-SF} & RI & 0.424	& 0.191 &  	0.817 &	0.755 &	0.806  &	0.892$^\dag$ & 0.818 &	0.765 &	0.746 &	0.763 &	0.789 &	0.726 &	0.858 & 0.842 & \textbf{0.876}     \\
                                           & NMI & 0.011	& 0.066 &  	0.491 &	0.402 &	0.425  &	0.614$^\dag$ & 0.495	& 0.290 &	0.102 & 0.278 &	0.331 &	0.026 &	0.537 & 0.528 &  \textbf{0.583} \\
\multirow{2}{*}{PenDigits} & RI	& 0.929 & 0.922 &  	0.935 &	0.881 &	0.885  &	0.946 & 0.914 &	0.941 &	0.936 &	0.873 &	0.857 &	0.818 & \textbf{0.950} &	0.921  & 0.932       \\
                                           & NMI & 0.713	& 0.693 &  	0.738 &	0.605 &	0.563  &	0.826$^\dag$ & 0.705 &	0.776 & 0.749 &	0.537 &	0.339 &	0.090 & \textbf{0.822} & 0.703  & 0.730  \\
\multirow{2}{*}{StandWalkJump} & RI & 0.591	& 0.762 &  	0.733 &	0.695 &	0.733  &	0.771$^\dag$ & 0.500 &	0.410 &	0.410 &	0.457 &	0.589 &	0.467 & 0.724  &	0.591  & \textbf{0.761}    \\
                                           & NMI & 0.350	& 0.536 &  	0.559 &	0.466 &	0.556  &	0.609$^\dag$ & 0.077 &	0.213 &	0.213 &	0.193 &	0.187 &	0.248 &	0.554 & 0.336 & \textbf{0.611}   \\
\multirow{2}{*}{UWaveGestureLibrary} & RI & 0.883	& 0.913 &  	0.920 &	0.893 &	0.879  &	0.913 & 0.863 &	0.865 &	0.893 &	0.817 &	0.796 &	0.779 &	0.927 &	0.927    &  \textbf{0.929}$^\dag$  \\
                                           & NMI & 0.570	& 0.710 &  	0.713 &	0.582 &	0.558  &	0.728 &	0.610 & 0.511 &	0.614 & 0.322 &	0.215 &	0.244 &	0.731 &	0.731 & \textbf{0.749}$^\dag$   \\
\hline     
\multirow{2}{*}{\textbf{Avg Cohen's d Effect}}            & RI & -0.451 	& -0.129 & 	0.287 &  -0.065	& -0.230   &1.096 &-0.012	 &-0.227	 &-0.111	 &-1.095  &-0.771	 &-0.816  & 1.084 &  0.321 & \textbf{1.119}$^\dag$ \\
& NMI	& -0.244 & 0.000 & 	0.217	& 0.004	   & -0.316		& 1.152$^\dag$	& 0.153	& -0.265	& -0.232	& -1.302	&-0.910	& -0.892 & 1.071 & 0.478 & \textbf{1.089}   \\
                                           
\hline
\textbf{Avg Ranking } (URL only) & & / & / & / & / & / &
        / & / &  5.08 & 5.25 &  6.92 & 6.08    &  5.92
       & 1.75 & 3.25 & \textbf{1.33}$^\dag$    \\
\textbf{Avg Ranking } (All) & & 9.92 & 7.83 & 6.75 & 8.50 & 9.42 & 2.92 & 8.17 & 9.25 & 8.92 & 12.67 & 11.25 & 11.67  & 2.66 & 6.33 & \textbf{2.42}$^\dag$ \\
    \bottomrule
    \end{tabular}}
\end{table*}

\begin{table}[t]
        \centering
        \caption{Comparison of performance of on UCR anomaly detection dataset. The sliding window's length is set to 256, with the most optimal outcomes highlighted in bold. }
        \label{tab:ad_ucr}
        \resizebox{\linewidth}{!}{
        \begin{tabular}{lccccccc}
        \toprule
         \textbf{Methods on UCR} & IF-s & \method{TST}{zerveas2021transformer} 
& \method{TS-TCC}{eldele2021time} 
& \method{T-Loss}{franceschi2019unsupervised} 
& \method{TS2Vec}{yue2022ts2vec} 
& \method{CSL}{liang2023contrastive} &\textbf{MMFA}  \\
         \midrule


Precision 
&0.9778
&0.7598
&0.7863
&0.7986
&0.8147
&0.8040
&0.9218 \\

Recall 
&0.0069
&0.3481
&0.3431
&0.3556
&0.3551
&0.3664
&0.4717 \\


F1  
&0.0137
&0.4775
&0.4777
&0.4921
&0.4946
&0.5034
&\textbf{0.6241}\\

Accuracy  
&0.0069
&0.9881
&0.9877
&0.9882
&0.9881
&0.9886
&\textbf{0.9920} \\
 

          \bottomrule
        \end{tabular}
        }
\end{table}

% \begin{table}[t]
%         \centering
%         \caption{Comparison of performance in anomaly detection of MTS on SMAP, MSL, SMD, and ASD datasets, evaluated on F1 scores. The sliding window's length is set to 100, with the most optimal outcomes highlighted in bold. }
%         \label{tab:anomaly_detection}
%         \resizebox{1.08\linewidth}{!}{
%         \begin{tabular}{lccccccccccc}
%         \toprule
%          \textbf{Dataset} & IF-s & IF-p & \method{TS2Vec}{yue2022ts2vec} & 
% \method{T-Loss}{franceschi2019unsupervised} & 
% \method{TNC}{tonekaboni2021unsupervised} & 
% \method{TS-TCC}{eldele2021time} & 
% \method{TST}{zerveas2021transformer} & 
% \method{CSL}{liang2023contrastive} & MMFA-aug & \textbf{MMFA} \\
%          \midrule
% SMAP  
% &0.0890
% &0.2166
% &0.2834
% &0.3862
% &0.3218
% &0.3501
% &0.2511
% &0.4049
% &0.2654
% &\textbf{0.4142}\\
% MSL 
% &0.0077
% &0.2653
% &0.1753
% &0.2771
% &0.2629
% &0.3074
% &0.2645
% &\textbf{0.4033}
% &0.3325
% &0.3923\\
% SMD 
% &0.2912
% &0.2152
% &0.2846
% &0.2472
% &0.2291
% &0.2454
% &0.1859
% &0.2784
% &0.3525
% &\textbf{0.4027}\\
% ASD  
% &0.2585
% &0.3193
% &0.3735
% &0.3449
% &0.2945
% &0.2679
% &0.2926
% &0.4349
% &0.3722
% &\textbf{0.4626}\\
% \hline
% \textbf{Avg Cohen's d Effect}  
% &-1.600 
% &-0.589 
% &-0.127 
% &0.188 
% &-0.321 
% &-0.153 
% &-0.718
% &1.080
% &0.529
% &\textbf{1.711}\\ 

% \hline
% \textbf{Avg Ranking} & 8.25 & 7.50 & 5.50 & 4.75 & 7.00 & 6.00 & 8.25 & 2.50 & 4.00 & \textbf{1.25}\\
%           \bottomrule
%         \end{tabular}
%         }
% \end{table}
\begin{researchq}
How does the MMFA framework perform compared to other unsupervised methods?
\end{researchq}

\textbf{Tab.}~\ref{tab:classification}, \textbf{Tab.}~\ref{tab:clustering}, and \textbf{Tab.}~\ref{tab:ad_ucr} show the comparison results for classification, clustering, and anomaly detection tasks. The proposed MMFA framework consistently outperforms unsupervised competitors across most tasks and datasets, demonstrating superior overall performance. Specifically, MMFA achieves the highest accuracy in 21 out of 30 datasets, ranking the best among all baselines. Moreover, MMFA surpasses all unsupervised methods in total dataset wins, aligning with expectations due to its capability to inject cross-modal knowledge into the time series encoder, leveraging diverse feature engineering techniques.

For a fair comparison, we crafted a variant of MMFA, denoted as MMFA-aug, solely employing data augmentations: jittering, cropping, warping, and quantizing pooling, with the same time series encoder. MMFA's performance significantly surpassed MMFA-aug, due to denoising effects and the extraction of salient patterns by various transforms and neural encoders, as detailed in \textbf{Sec.}~\ref{sec:algo}. This integration empowered MMFA to induce more extensive discriminative patterns.

The clustering results are displayed in \textbf{Tab.}~\ref{tab:clustering}. MMFA surpasses all competitors except for PenDigits and AtrialFibrillation datasets. For PenDigits (length 8), MMFA demonstrates limitations in learning from short time series, as transforms struggle to capture significant patterns.

The anomaly detection results are shown in \textbf{Tab.}~\ref{tab:ad_ucr}. Compared to other time series representation methods, MMFA always shows a large margin over them. This shows that MMFA creates a more compact and detailed boundary of training distribution, by shrinking the arrangement of representation learned over arrangements of multi-modal features.


Taking advantage of multi-modal feature transforms, MMFA surpasses simple data augmentation-based methods~\cite{yue2022ts2vec,liang2023contrastive,eldele2021time,zerveas2021transformer} compared in \textbf{Tab.}~\ref{tab:classification}, \textbf{Tab.}~\ref{tab:clustering} and \textbf{Tab.}~\ref{tab:ad_ucr}. It induces global data structures by learning from algorithmic bias deduced from other transform-encoder compositions, yielding robust representations.

\begin{researchq}
How does the MMFA framework perform compared to the supervised approaches and approaches tailored for the downstream tasks?
\end{researchq}

The comparison between MMFA and tailored discriminative models on classification tasks is shown in \textbf{Tab.}~\ref{tab:classification}. MMFA outperforms tailored fully supervised competitors for time series classification on 16 UCR classification datasets, showing significantly higher improvement over all baselines according to Cohen's d effect, indicating its extraction of richer information directly from raw time series during inference.

According to \textbf{Tab.}~\ref{tab:clustering}, over all of the clustering tasks, among the tailored methods, only MUSLA~\cite{zhang2022multiview} is compatible with MMFA. MMFA also shows significant advantages among the URL methods.

In contrast to the compared tailored methods in \textbf{Tab.}~\ref{tab:classification} and \textbf{Tab.}~\ref{tab:clustering}, unsupervised methods learn to transform properties of data without task-specific tuning, enhancing feature extraction capabilities that go beyond specific tasks, and fostering potential generalizability. These methods are label-free, automating the discovery of latent structures and relationships within data, and revealing insights not apparent in supervised techniques reliant on predefined labels. In \textbf{Sec.}~\ref{sec:algo}, we showcase MMFA's capacity to discern patterns, significantly contributing to uncovering hidden structures and providing novel data perspectives.

For further evaluations and research questions, we statistically analyzed MMFA's performance on datasets with different characteristics (\textbf{RQ3}), comparisons between pre-trained and non-pre-trained encoders (\textbf{RQ4}), introduced an ablation (Ab) and leave-one-out (LOO) studies (\textbf{RQ5}) and sensitivity analysis (\textbf{RQ6}), in the appendix, \textbf{Sec.}~\ref{sec:additional_RQs}, presenting a more comprehensive analysis of the proposed methods.

\section{Conclusion and Future Directions}\label{sec:conc}

In this study, we propose a framework aimed at mitigating the limitations arising from intrinsic feature engineering in Universal Representation Learning (URL) for Multivariate Time Series (MTS). The incorporation of feature alignment and regularization methods forms a revisitation of the spectral selection mechanism for constructing URL framework, surpassing existing state-of-the-art approaches.

Looking ahead, this study unveils several promising avenues. For instance, delving deeper into transform property deduction and more efficient learning methods could yield more comprehensive representations. The integration of broader external knowledge or domain-specific information holds the potential to further enrich these representations and accelerate the learning process.


% In the unusual situation where you want a paper to appear in the
% references without citing it in the main text, use \nocite
% \nocite{langley00}

\bibliography{example_paper}
\bibliographystyle{icml2025}


%%%%%%%%%%%%%%%%%%%%%%%%%%%%%%%%%%%%%%%%%%%%%%%%%%%%%%%%%%%%%%%%%%%%%%%%%%%%%%%
%%%%%%%%%%%%%%%%%%%%%%%%%%%%%%%%%%%%%%%%%%%%%%%%%%%%%%%%%%%%%%%%%%%%%%%%%%%%%%%
% APPENDIX
%%%%%%%%%%%%%%%%%%%%%%%%%%%%%%%%%%%%%%%%%%%%%%%%%%%%%%%%%%%%%%%%%%%%%%%%%%%%%%%
%%%%%%%%%%%%%%%%%%%%%%%%%%%%%%%%%%%%%%%%%%%%%%%%%%%%%%%%%%%%%%%%%%%%%%%%%%%%%%%
\newpage
\appendix
\onecolumn

\section{Related Work}\label{sec:rela}

In this section, we overview the research in the landscape related to the domain of our work. We categorize them into three themes and discuss their contributions, limitations, and relevance to our work.

\subsection{Representation Learning for Time Series}

Unsupervised representation learning provides a scalable approach to obtain meaningful representations~\cite{meng2023unsupervised}. Neural architectures such as autoencoders and seq2seq models~\cite{vadiraja2020survey} have paved the way for methods based on data reconstruction and context prediction to learn representations~\cite{ma2019learning, malhotra2017timenet}. Contrastive learning distinguishes the strict positive (augmented by the same sample) and negative (positive) sample pairs to capture invariance after augmentation, which is presented as an important method~\cite{huynh2022boosting,franceschi2019unsupervised, eldele2021time, yue2022ts2vec, wu2022timesnet}. Representation learning not only unifies different tasks but also provides scalable ways for time series data management~\cite{paparrizos2019grail}.



Notably, URL encompasses multi-levels of learning objectives: instance level~\cite{chen2020big, oord2018representation,chen2020simple}, cluster (prototype) level~\cite{li2020prototypical, caron2020unsupervised, meng2023mhccl}, and temporal level~\cite{tonekaboni2021unsupervised, eldele2021time, hyvarinen2016unsupervised}, etc., each capturing different aspects of similarity and dependency in data, representing preservation of \textit{local} or \textit{global} arrangement of the temporal data.

On time series, instance-level contrastive learning perturbs original samples, emphasizing the retention of salient features~\cite{yue2022ts2vec}. In this case, the proposed supervision signal enhances the preservation of the \textit{local} arrangement of data sampled from a distribution of permuted samples. It works well when downstream task goals are sensitive to \textit{local} fluctuations. 

Cluster-level contrastive learning extends the focus to the \textit{global} shape of sample distributions~\cite{meng2023mhccl}. Meanwhile, temporal-level contrastive learning captures temporal dependencies surrounding each sample, learning the causal structure of time series data as \textit{global} arrangement~\cite{eldele2021time}. GRAIL~\cite{paparrizos2019grail} learns temporal representation vectors to reconstruct the distance matrix with the temporal invariance distance measure. These methods \textit{globally} align \textit{local} structures captured naturally by the encoder, thus, injecting different global information into the feature encoder.

\subsection{Feature Transforms}\label{sec:transform_main}

Researchers have investigated various signal processing and feature engineering techniques. These techniques help preserve salient patterns that are challenging to capture without additional prior knowledge. Here's how these techniques achieve this and their impact. In this paper, we leverage four of them. (1). Discrete Fourier transform (DFT)~\cite{yang2022unsupervised,winograd1978computing}, which reveals the spectral structure of the original time series. (2). Continuous wavelet transform (CWT)~\cite{grossmann1990reading} can help to balance the temporal and spectral resolution. (3). image-like features (i.e., GAF, RP~\cite{wang2015encoding}), which can be encoded leveraging the local precentral view of CNNs. (4). Symbolic features (i.e., SFA~\cite{tang2020interpretable}, SAX~\cite{notaristefano2013data}), which view the time series as the composition of symbols that can be semantically computed and understood. Full details about these transforms are elaborated in the appendix, \textbf{Sec.}~\ref{sec:transforms}. 

Generally, these methods provide sparse and overcomplete representations that are challenging to fully take advantage of in various tasks. Our proposed framework can make a dense representation of them comprehensively and efficiently.


\subsection{URL Based on Regularization and Multi-view Learning}

Despite successful contrastive learning, the trend leans toward methods integrating direct sample binding and alignment in URL, and learning hypotheses from multiple views of grounded operations, properties, and concepts~\cite{girdhar2023imagebind, yariv2023audiotoken}. Regularization-based approaches~\cite{bardes2021vicreg} focus on positive sample pairs to reduce Dirichlet energy on a graph, bypassing the curse of dimensionality linked to numerous negative samplings~\cite{balestriero2022contrastive}.

Existing methods leveraging multi-modal features often rely on feature fusion \cite{yang2022unsupervised, tang2020interpretable}. However, in URL for time series, these approaches, exemplified by \cite{yang2022unsupervised}, focus on domain-specific transforms and fusion strategies without considering informative associations across views. 

Our algorithm learns global and local sample arrangements effectively, resulting in stronger theoretical guarantees and outperforming state-of-the-art URL methods.

\section{Feature transforms}\label{sec:transforms}

In this section, we provide detailed descriptions of the transforms leveraged in the proposed method, which are first introduced briefly in \textbf{Sec.}~\ref{sec:transform_main}.

\subsection{Discrete Fourier Transform (DFT)}

With the same neural architectures and similar training objectives, URL methods using only DFT transformed time series or only raw data share a small proportion of false prediction during evaluation, with the rest of them being non-overlapping, revealing relatively independent inductive bias~\cite{yang2022unsupervised}. 

Fourier transform decomposes the time series into its frequency components~\cite{winograd1978computing}, while raw time series data contains information about the temporal trends and fluctuations.


The transformed and raw data offer complementary insights into underlying patterns. False predictions often stem from distinct false patterns introduced respectably. Merging these views amalgamates their complementary information.

\subsection{Continuous wavelet transform (CWT)}

CWT is a tool that provides an overcomplete representation of a signal by letting the translation and scale parameter of wavelets vary continuously~\cite{grossmann1990reading}. The abundance of features generated by various mother wavelets and overcompleteness allow analysis with higher accuracy. 

Compared to raw time series that collapse spectral features, or spectral transforms obscuring temporal stages and trends, Continuous Wavelet Transform (CWT) arranges temporal and spectral patterns in a 2D plane, generating informative yet redundant features for URL. However, its high computational time for Multivariate Time Series (MTS) limits its use in time series URL studies.

\subsection{Encoding Time Series To Image}\label{sec:imag}

\cite{wang2015encoding} proposes a framework to encode time series data as different types of "images" thus allowing machines to "visually" recognize and classify time series. This approach helps to extract patterns and structures that are less identifiable in the raw data. For example, using a polar coordinate system, Gramian Angular Field (GAF) images a represented as a Gramian matrix where each element is the trigonometric sum between different time intervals.   

Learning methods based on these transforms always leverage a CNN structure to exploit translational invariance within the "images" by extracting features through receptive fields.

\subsection{Symbolic transform}

Symbolic representation techniques like Symbolic Aggregate Approximation (SAX)\cite{notaristefano2013data} or Symbolic Fourier Approximation (SFA)\cite{schafer2012sfa} transform noisy time series data into abstract symbolic patterns, filtering out noise and reducing classifier overfitting.

SFA, a promising method for time series classification~\cite{tang2020interpretable}, aids in handling high-dimensional, sparse data prone to overfitting. Our approach utilizes these transforms to construct multi-modal views of raw data, guiding the raw MTS encoder to match distributions of crucial patterns extracted by diverse transform-encoder compositions, addressing challenges related to high dimensionality, sparsity, and complex pattern interactions in time series data.

\section{Neural Encoder}\label{sec:neur_encoder}

In this section, we introduce the rest of the encoders for all the modalities transformed from raw time series, and their domains and codomains, which are first mentioned in \textbf{Sec.}~\ref{sec:neur_encoder_main}.

\subsection{Convolutional Neural Network}

To extract patterns from spectral sequences and two-dimensional matrices using local perceptive fields, we utilize one and two-dimensional ResNet CNN architectures as feature extractors. $T^{(img)}: \mathbb{R}^{D\times T} \to \mathbb{R}^{D \times d_w \times d_h}$ represents the transform operator encoding raw MTS samples into $D$ 2D image-style feature matrices (\textbf{Sec.}~\ref{sec:imag}). Similarly, $T^{(cwt)}$ represents the CWT operator generating 2D feature matrices for $D$ channels. Both are accompanied by 2D ResNet encoders, which also include interpolation for ensuring the encoder's input size $d_w \times d_h$. As for the DFT transform operator, $T^{(dft)}: \mathbb{R}^{D\times T} \to \mathbb{R}^{D \times T}$, we employ a 1D ResNet as the encoder.

\subsection{Transformer}

To capture patterns that imply symbolic features transformed from the raw time series. We take advantage of the methods that are used in sequence modeling of natural language. Transformers can be used as feature extractors, and importantly, they can be pre-trained to enhance diverse information. We observe a boost in the downstream task performance with a pre-trained transformer as a language model. 

The symbolic transformer operator $T^{(sfa)}: \mathbb{R}^{D\times T} \to \mathbb{R}^{L \times d_e}$ first transforms MTS to token sequences, then the joint the sequences to a long sequence with separators. $L$ denotes the length of the token sequence. Finally, we look up the word embeddings of the tokens as input of the transformer encoder.



\section{Experiment Details}\label{sec:exp_detail}

In this section, we exhibit the details of implementations of downstream tasks and baselines, characteristics of the datasets, evaluation metrics, and implementation details of our proposed methods, which are sufficient support for our demonstration of the experiments in \textbf{Sec.}~\ref{sec:empi}. 


\subsection{Experimental Settings}
We perform extensive experiments on a total of 31 real-world datasets to comprehensively analyze the quality of MMFA representations across diverse patterns. Our investigation encompasses three primary tasks, supervised classification, unsupervised clustering, and anomaly detection. It is noteworthy that in the case of anomaly detection, the MTS representation is considered at the segment level (as opposed to observation-level \cite{li2021multivariate, su2019robust}). 

Specifically, we follow the protocols introduced in \cite{Wu2020CurrentTS} to evaluate accuracies and F1 scores for the UCR anomaly datasets. To address these tasks, we train famous basic models, i.e., SVM, K-means, and Isolation Forest on the acquired representations, w.r.t the classification clustering and anomaly detection task. The datasets, baseline methods, implementations, and evaluation metrics are presented for clarity.

\subsubsection{Datasets} To assess the representation quality across three downstream tasks, we employ 35 diverse MTS datasets. These datasets exhibit variations in sample size, dimensionality, length, number of classes, and application scenarios. The default train/test split is applied uniformly across all datasets, with the encoder and task-specific models exclusively trained on the training samples. The specific datasets utilized for each task are outlined below. The original datasets' dimensionalities (channel numbers) can overwhelm 2D transforms (e.g., RP, CWT) and ResNet encoders due to limited computation resources. Therefore, we perform average pooling on transformed multi-channel 2D features, capping dataset channels at 64. Raw time series channels remain unchanged for input into the time series encoder.




\textit{\underline{Classification.}} We assess the performance of MTS categorization across all 30 datasets from the widely used UEA archive~\cite{bagnall2018uea}. These datasets encompass diverse domains such as human action recognition, Electrocardiography monitoring, and audio classification~\cite{bagnall2018uea}. The statistical details of the datasets can be found in \textbf{Tab.}~\ref{tab:uea_statistics}.

\textit{\underline{Clustering.}} In line with a recent study on clustering multivariate time series, we assess the performance of clustering using 12 diverse UEA subsets. These subsets exhibit significant heterogeneity in terms of training/test set sizes, length, as well as the number of dimensions and classes. The statistics for these 12 datasets are shown in \textbf{Tab.}~\ref{tab:uea_statistics} (denoted by *).


\textit{\underline{Anomaly Detection.}} In this study, we leverage four recently released datasets sourced from diverse real-world applications to perform anomaly detection. The datasets include anomaly data from the Soil Moisture Active Passive satellite (SMAP) and the Mares Science Laboratory rover (MSL), both obtained through NASA~\cite{hundman2018detecting}. Additionally, we utilize the Server Machine Data (SMD), a dataset spanning five weeks, collected by~\cite{su2019robust} from a major Internet company. The Application Server Dataset (ASD) covers 45 days and characterizes the server's status, recently compiled by~\cite{li2021multivariate}. Following the methodology outlined in~\cite{li2021multivariate}, we evaluate SMD using 12 entities unaffected by concept drift. The dataset statistics are shown in \textbf{Tab.}~\ref{tab:ad_statistics}. We also use the UCR Anomaly Detection Dataset~\cite{Wu2020CurrentTS}, which aggregates 250-time series data sub-sets across multiple domains for robust anomaly detection testing.

\subsubsection{Baselines} We use 21 baselines for comparison, which are divided into two groups:

\textit{\underline{URL methods.}} We compare our MMFA framework with 6-time series URL baselines, including TS2Vec~\cite{yue2022ts2vec}, T-Loss~\cite{franceschi2019unsupervised}, TNC~\cite{tonekaboni2021unsupervised}, TS-TCC~\cite{eldele2021time},  TST~\cite{zerveas2021transformer} and CSL~\cite{liang2023contrastive}. All URL competitors are evaluated similarly to MMFA for a fair comparison.

\textit{\underline{Methods tailored to specific tasks.}} We also incorporate benchmarks customized for downstream tasks. We opt for distinguished strategies in classification, including the widely-used baseline DTWD~\cite{bagnall2018uea}. DTWD employs a one-nearest-neighbor classifier with dynamic time warping as the distance metric. Additionally, we consider five supervised techniques: MLSTM-FCNs~\cite{karim2019multivariate} utilizing recurrent neural networks, TapNet~\cite{zhang2020tapnet} employing attentional prototypes, ShapeNet~\cite{li2021shapenet} based on shapelets, and CNN-based models OSCNN~\cite{tang2020omni} and DSN~\cite{xiao2022dynamic}. We exclude ensemble methods such as those outlined in~\cite{lines2018time} to ensure a fair comparison. It is worth noting that the supervised classification methods leverage true labels for feature learning, akin to data augmentation or sampling in URL. Hence, the fairness of the comparison between MMFA and the baselines is maintained.

We assess six sophisticated clustering benchmarks, including feature selection based Time2Feat~\cite{bonifati2022time2feat}, dimension-reduction-based MC2PCA~\cite{li2019multivariate} and TCK~\cite{mikalsen2018time}, distance-based m-kAVG+ED and m-kDBA~\cite{ozer2020discovering}, deep learning-based DeTSEC~\cite{ienco2020deep}, and shapelet-based MUSLA~\cite{zhang2022multiview}.

Since no documented anomaly detection evaluations in segment-level settings exist, we create two raw MTS baselines using Isolation Forest for fair comparisons, models operating at each timestamp (IF-p) or within each sliding window (denoted as IF-s).

\subsubsection{Evaluation Metrics.} Standard metrics are used to assess downstream task performance. Accuracy (Acc)~\cite{bagnall2018uea} is applied for classification tasks. Clustering outcomes are measured using Rand Index (RI) and Normalized Mutual Information (NMI)~\cite{zhang2022multiview, zhang2018salient}. Anomaly detection employs F1-score~\cite{li2021multivariate}.

In this study, we employed Cohen's d~\cite{becker2000effect} as a metric to analyze the performance enhancement of our proposed algorithm relative to existing methods across multiple datasets. The average effect size for the performance improvement of the $i$th algorithm over the others was calculated using the formula \( c_i = \frac 1 {n-1} \sum_{j\neq i}\frac{p_i - p_j}{ std(p_1, \ldots, p_n)} \), where \( p_i, i \in [n] \) represent the performance of the respective algorithms. The standard deviation is computed from the performance values across all methods. By evaluating Cohen's d values for each method, we derived an average effect size that encapsulates the overall performance improvement of our algorithm in comparison to the benchmark methods.

\subsection{Implementation Details.} The MMFA framework is implemented using PyTorch 1.10.2, and all experiments run on a Ubuntu machine equipped with Tesla A100 GPUs. Data augmentation methods are implemented using tsaug~\cite{tsaug} with default parameters.

The majority of MMFA's hyperparameters are consistently assigned fixed values across all experiments, devoid of any hyperparameter optimization. The coefficients $\alpha$ and $\beta$ in Eq.~\ref{eq:loss} are both assigned a value of 25, while $\gamma$ is set to 1. Additionally, $\epsilon$ takes on the value of $10^{-7}$ in Eq.~\ref{eq:cov}. The SGD optimizer is employed to train all feature encoders with a learning rate fixed at $10^{-4}$.

For simplicity, to discover high-performance transform-encoder compositions for each of the datasets, we treat the raw time series encoder as the \textbf{main encoder}, and evaluate the rest of each of the compositions in \textbf{Tab.}~\ref{tab:TFpairs}, respectively. Then, the main encoder and the rest of the top-3 encoders are trained together according to Alg.~\ref{algo}. 

Finally, we report the highest performance. For all datasets, the batch size is uniformly set to 8. We reproduce the time series URL baseline by employing the publicly available code provided by the original authors, configured as recommended. The classification baselines and task-specific clustering baselines' outcomes are extracted from the cited publications~\cite{bagnall2018uea, li2021shapenet, tang2020omni, xiao2022dynamic, yue2022ts2vec, zhang2022multiview}. Our reproduced results cover other aspects.


\begin{table}[h]
    \centering
    \caption{Statistics on the 30 UEA datasets are employed for classification assessment, while 12 specific subsets (denoted by $^*$) undergo clustering evaluation as per the methodology outlined in ~\cite{zhang2022multiview}.}
    \resizebox{.55\linewidth}{!}{\begin{tabular}{lccccc}
\toprule
Dataset	& \# Train &	\# Test &	\# Dim &	Length &	\# Class \\
\midrule

ArticularyWordRecognition$^*$ & 275 & 300 & 9 & 144 & 25 \\

AtrialFibrillation$^*$ & 15 & 15 & 2 & 640 & 3 \\

BasicMotions$^*$ & 40 & 40 & 6 & 100 & 4 \\

CharacterTrajectories & 1422 & 1436 & 3 & 182 & 20 \\

Cricket & 108 & 72 & 6 & 1197 & 12 \\

DuckDuckGeese & 50 & 50 & 1345 & 270 & 5 \\

EigenWorms & 128 & 131 & 6 & 17984 & 5 \\

Epilepsy$^*$ & 137 & 138 & 3 & 206 & 4 \\

EthanolConcentration & 261 & 263 & 3 & 1751 & 4 \\

ERing$^*$ & 30 & 270 & 4 & 65 & 6 \\

FaceDetection & 5890 & 3524 & 144 & 62 & 2 \\

FingerMovements & 316 & 100 & 28 & 50 & 2 \\

HandMovementDirection$^*$ & 160 & 74 & 10 & 400 & \textit{4} \\

Handwriting & 150 & 850 & 3 & 152 & 26 \\

Heartbeat & 204 & 205 & 61 & 405 & 2 \\

InsectWingbeat & 30000 & 20000 & 200 & 30 & 10 \\

JapaneseVowels & 270 & 370 & 12 & 29 & 9 \\

Libras$^*$ & 180 & 180 & 2 & 45 & 15 \\

LSST & 2459 & 2466 & 6 & 36 & 14 \\

MotorImagery & 278 & 100 & 64 & 3000 & 2 \\

NATOPS$^*$ & 180 & 180 & 24 & 51 & 6 \\

PenDigits$^*$ & 7494 & 3498 & 2 & 8 & 10 \\

PEMS-SF$^*$ & 267 & 173 & 963 & 144 & 7 \\

Phoneme & 3315 & 3353 & 11 & 217 & 39 \\

RacketSports & 151 & 152 & 6 & 30 & 4 \\

SelfRegulationSCP1 & 268 & 293 & 6 & 896 & 2 \\

SelfRegulationSCP2 & 200 & 180 & 7 & 1152 & 2 \\

SpokenArabicDigits & 6599 & 2199 & 13 & 93 & 10 \\

StandWalkJump$^*$ & 12 & 15 & 4 & 2500 & 3 \\

UWaveGestureLibrary$^*$ & 120 & 320 & 3 & 315 & 8 \\


\bottomrule
    \end{tabular}}
    \label{tab:uea_statistics}
\end{table}


\begin{table}[h]
    \centering
    \caption{Statistics of evaluated anomaly detection datasets.}
    \resizebox{0.50\linewidth}{!}{\begin{tabular}{lccccc}
\toprule
Dataset	&  \# Entity  & \# Dim & Train length & Test length & Anomaly ratio (\%) \\
\midrule

UCR & 250 & 1 & 5302449 & 12919799 & 0.38 \\
\bottomrule
    \end{tabular}}
    \label{tab:ad_statistics}
\end{table}

\subsection{Additional Research Questions}\label{sec:additional_RQs}






\begin{researchq}
What is the impact of data characteristics?
\end{researchq}

In our analysis, the evaluated datasets exhibit significant variations in training dataset sizes, numbers of channels, time series length, and salient patterns. \textbf{Tab.}~\ref{tab:uea_statistics} illustrates the diversity: training set sizes range from 12 (StandWalkJump, comprising 3 classes with only 4 samples per class, constituting a 3-way, 4-shot learning task) to 30,000 (InsectWingbeat, characterized by abundant training samples). Consequently, we conduct correlation analysis on the UEA datasets focusing on these three data characteristics and show the findings in \textbf{Fig.}~\ref{fig:data_char}.

As depicted in (a) of \textbf{Fig.}~\ref{fig:data_char}, MMFA exhibits a high effect size on performance improvement with smaller training sets. This observation suggests that MMFA excels in achieving higher few-shot performance.

A slight negative correlation exists between classification performance improvement effect size and dataset dimensionality, which shows MMFA outperforms more significantly on low dimensional datasets, showcasing MMFA's strong generalization effects that avoid overfitting with certain patterns with limited input features.

Conversely, positive correlations between effect size and time series length indicate MMFA's stronger adaptability to longer time series than baselines. However, a limitation is that MMFA struggles with shorter time series (e.g., PenDigits datasets with $\text{length} = 8$), evident in lower performance across classification and clustering tasks. This limitation may stem from the transforms nature, requiring optimal performance from a time series of specific lengths.

\begin{figure}
    \centering
    \includegraphics[width=0.5\linewidth]{rala.pdf}
    \caption{\textbf{Correlation plot} illustrating the relationships between Cohen's d effect sizes of performance improvement made by MMFA on 30 UEA datasets and characteristics of the datasets, i.e., training size, time series dimensionality, and time length. Logarithmic transforms are employed to enhance the linearity of the data.}
    \label{fig:data_char}
\end{figure}

\begin{figure}
    \centering
    \includegraphics[width=0.6\linewidth]{sfa.pdf}\label{img:sfa}
    \caption{\textbf{Test set performance of pre-trained Longformer v.s Longformer trained from scratch.} In addition to raw time series and its encoder, we only choose SFA and pre-trained or non-pre-trained Longformer as the composition for transform and neural encoder, which are demonstrated in~\ref{tab:TFpairs} on SelfRegulationSCP2 and UWaveGestureLibrary dataset. It is shown in the figure that pre-trained Longformer finally gains much higher performance.}
    \label{fig:sfa}
\end{figure}

\begin{researchq}
How does the utilization of pretraining impact unsupervised time series representation learning compared to non-pre-trained models?
\end{researchq}

We conducted experiments comparing the performance of a pre-trained Longformer model against a non-pre-trained one. Initially, the time series data was transformed into symbolic word bags. Our approach aimed to assess how pretraining via the Word-Word Masking (WWM) task, influenced unsupervised time series representation learning.

The findings show that pretraining finally boosts performance compared to non-pre-trained models. Despite the slower convergence of the pre-trained Longformer model shown in \textbf{Fig.}~\ref{fig:sfa}, it ultimately outperformed the non-pre-trained version by a wide margin. This performance boost is probably due to the BERT encoder's capacity to identify patterns between the WWM-pre-trained corpus and symbolic features from time series data. Overall, these results highlight the value of pretraining in enhancing the extraction of meaningful representations from time series data.


\begin{researchq}
    Ablation (Ab) and leave one out (LOO) study.
\end{researchq}
According to Table~\ref{ablation}, more transformations lead to better performance and lower variation in the magnitude of the overall effect. This is the evidence for the robustness of the model.

\begin{table}[h]
    \centering
    \caption{Ablation (Ab) and leave one out (LOO) study.}\label{ablation}
    \resizebox{0.8\linewidth}{!}{\begin{tabular}{lcccccccccc}
    \toprule
    Dataset & CGau2 & RP & GADF & DFT & WEASEL & CGau2-loo & RP-loo & GADF-loo & DFT-loo & WEASEL-loo \\
    \midrule
    ArticularyWordRecognition & 0.983 & 0.983 & 0.980 & 0.980 & 0.980 & 0.983 & 0.980 & 0.983 & 0.987 & 0.983 \\
    AtrialFibrillation & 0.533 & 0.533 & 0.533 & 0.467 & 0.467 & 0.533 & 0.533 & 0.533 & 0.533 & 0.533 \\
    BasicMotions & 1.000 & 1.000 & 1.000 & 1.000 & 1.000 & 1.000 & 1.000 & 1.000 & 1.000 & 1.000 \\
    CharacterTrajectories & 0.991 & 0.987 & 0.987 & 0.991 & 0.989 & 0.990 & 0.991 & 0.988 & 0.990 & 0.991 \\
    Cricket & 0.958 & 0.944 & 0.958 & 0.944 & 0.958 & 0.972 & 0.972 & 0.944 & 0.958 & 0.972  \\
    DuckDuckGeese & 0.300 & 0.360 & 0.340 & 0.340 & 0.360 & 0.320 & 0.300 & 0.280 & 0.340 & 0.460 \\
    EigenWorms & 0.580 & 0.550 & 0.557 & 0.595 & 0.557 & 0.626 & 0.626 & 0.588 & 0.580 & 0.580 \\
    Epilepsy & 0.964 & 0.942 & 0.978 & 0.957 & 0.964 & 0.957 & 0.971 & 0.971 & 0.964 & 0.971 \\
    EthanolConcentration & 0.449 & 0.479 & 0.479 & 0.475 & 0.426 & 0.616 & 0.620 & 0.631 & 0.601 & 0.643 \\
    ERing & 0.956 & 0.959 & 0.952 & 0.956 & 0.956 & 0.952 & 0.948 & 0.948 & 0.944 & 0.956 \\
    FaceDetection & 0.564 & 0.570 & 0.562 & 0.564 & 0.576 & 0.575 & 0.573 & 0.567 & 0.574 & 0.568 \\
    FingerMovements & 0.510 & 0.490 & 0.520 & 0.540 & 0.530 & 0.580 & 0.590 & 0.590 & 0.570 & 0.550 \\
    HandMovementDirection & 0.405 & 0.432 & 0.432 & 0.432 & 0.446 & 0.446 & 0.392 & 0.419 & 0.432 & 0.486 \\
    Handwriting & 0.487 & 0.500 & 0.496 & 0.492 & 0.489 & 0.474 & 0.485 & 0.476 & 0.481 & 0.487 \\
    Heartbeat & 0.722 & 0.741 & 0.722 & 0.732 & 0.737 & 0.707 & 0.722 & 0.741 & 0.732 & 0.722 \\
    InsectWingbeat & 0.262 & 0.425 & 0.466 & 0.466 & 0.290 & 0.262 & 0.425 & 0.466 & 0.466 & 0.290 \\
    JapaneseVowels & 0.892 & 0.895 & 0.886 & 0.886 & 0.895 & 0.905 & 0.897 & 0.900 & 0.908 & 0.889 \\
    Libras & 0.872 & 0.872 & 0.872 & 0.872 & 0.872 & 0.889 & 0.878 & 0.883 & 0.894 & 0.883 \\
    LSST & 0.607 & 0.600   & 0.601 & 0.603 & 0.599 & 0.593 & 0.603 & 0.595 & 0.607 & 0.601 \\
    MotorImagery & 0.510 & 0.520 & 0.640 & 0.630 & 0.610 & 0.620 & 0.590 & 0.600 & 0.590 & 0.590 \\
    NATOPS & 0.817 & 0.839 & 0.833 & 0.822 & 0.817 & 0.856 & 0.811 & 0.844 & 0.828 & 0.828 \\
    PenDigits & 0.984 & 0.985 & 0.985 & 0.985 & 0.986 & 0.984 & 0.985 & 0.986 & 0.986 & 0.985 \\
    PEMS-SF & 0.809 & 0.798 & 0.798 & 0.803 & 0.798 & 0.803 & 0.809 & 0.809 & 0.809 & 0.809 \\
    PhonemeSpectra & 0.241 & 0.236 & 0.245 & 0.236 & 0.245 & 0.233 & 0.222 & 0.228 & 0.232 & 0.225 \\
    RacketSports & 0.855 & 0.862 & 0.855 & 0.882 & 0.855 & 0.895 & 0.888 & 0.901 & 0.901 & 0.888 \\
    SelfRegulationSCP1 & 0.782 & 0.785 & 0.785 & 0.785 & 0.782 & 0.819 & 0.816 & 0.812 & 0.816 & 0.795 \\
    SelfRegulationSCP2 & 0.467 & 0.539 & 0.472 & 0.472 & 0.456 & 0.506 & 0.506 & 0.517 & 0.500 & 0.483 \\
    SpokenArabicDigits & 0.980 & 0.983 & 0.981 & 0.977 & 0.978 & 0.984 & 0.983 & 0.980 & 0.985 & 0.985 \\
    StandWalkJump & 0.533 & 0.533 & 0.533 & 0.533 & 0.467 & 0.400 & 0.533 & 0.533 & 0.533 & 0.533 \\
    UWaveGestureLibrary & 0.912 & 0.916 & 0.912 & 0.916 & 0.916 & 0.912 & 0.912 & 0.916 & 0.906 & 0.916 \\
    \midrule
    \textbf{Average Effect Size} & -0.368 & -0.132 & -0.239 & -0.208 & -0.256 & 0.091 & 0.113 & 0.203 & 0.443 & 0.352 \\
    \bottomrule
    \end{tabular}}
    \label{tab:performance_metrics}
\end{table}
    

Transforms show non-additive effects. RP and the DFT show higher performance. RP contributes more independently to performance improvements. The insignificance of DFT in the LOO study suggests that it may be represented by or diminishing other jointly aligned transforms. CWT with CGau2 shows less significance but higher effects in the LOO study.

\begin{researchq}
Sensitivity analysis for $\alpha$, $\beta$ and $\gamma$ in the trainin objective.
\end{researchq}

According to \textbf{Tab.}~\ref{tab:alpha_sen} and \textbf{Tab.}~\ref{tab:beta_sen}, the accuracy demonstrates moderate sensitivity to parameter variations, with optimal $alpha$=10.0 for SelfRegulationSCP2 (0.544) and $alpha$=0.1 for Heartbeat (0.732).  For $\beta=\gamma$ parameters, both datasets achieve peak performance at $\beta=\gamma$=0.1/10.0 (Heartbeat: 0.732) and $\beta=\gamma$=10.0/25.0 (SelfRegulationSCP2: 0.544), suggesting parameter robustness in mid-to-high ranges, which encourages larger punishment of forbidding orthogonality of the representations.

\begin{table}[htbp]\label{tab:alpha_sen}
    \centering
    \caption{Accuracy comparison with different $\alpha$ values.}
    \label{tab:results}
    \resizebox{0.6\linewidth}{!}{\begin{tabular}{l *{5}{c}}
    \toprule
    Dataset & $\alpha=0.0$ & $\alpha=0.1$ & $\alpha=1.0$ & $\alpha=10.0$ & $\alpha=25.0$ \\
    \midrule
    SelfRegulationSCP2     & 0.528 & 0.528 & 0.528 & 0.544 & 0.533 \\
    Heartbeat & 0.712 & 0.732 & 0.712 & 0.722 & 0.712 \\
    \bottomrule
    \end{tabular}}
\end{table}

\begin{table}[htbp]\label{tab:beta_sen}
    \centering
    \caption{Accuracy comparison with different $\beta=\gamma$ values.}
    \label{tab:beta_results}
    \resizebox{0.7\linewidth}{!}{\begin{tabular}{l *{5}{c}}
    \toprule
    Dataset & $\beta=\gamma=0.0$ & $\beta=\gamma=0.1$ & $\beta=\gamma=1.0$ & $\beta=\gamma=10.0$ & $\beta=\gamma=25.0$ \\
    \midrule
    SelfRegulationSCP2       & 0.539 & 0.539 & 0.528 & 0.544 & 0.544 \\
    Heartbeat  & 0.717 & 0.732 & 0.722 & 0.732 & 0.722 \\
    \bottomrule
    \end{tabular}}
\end{table}




% \begin{researchq}
%     Computational overhead.
% \end{researchq}


% As a differentiable search process of a combination of transform properties as inductive bias, this could still be seen as effective compared to other search algorithms that can not benefit from parallelism.

% Backprop complexity with $B$ denoting individual encoding complexity.
% \[
% O \left( N \left( F_m + \sum_{i=1}^{k} F_i \right) + kND^2 + N \left( B_m + \sum_{i=1}^{k} B_i \right) \right)
% \]



% \begin{table}[h]
%     \centering
%     \caption{s/ep with varying \# of channels on DuckDuckGeese dataset.}
%     \resizebox{.4\linewidth}{!}{\begin{tabular}{lccccc}
%     \toprule
%     Model & 20 & 40 & 60 & 80 & 100 \\
%     \midrule
%     MMFA & 13.712 & 13.521 & 14.945 & 15.336 & 13.915 \\
%     TS2Vec & 2.844 & 3.196 & 2.717 & 3.445 & 3.502 \\
%     T-Loss & 10.516 & 9.490 & 10.659 & 10.635 & 10.212 \\
%     TS-TCC & 3.870 & 4.074 & 3.397 & 3.479 & 4.956 \\
%     TST & 0.755 & 0.883 & 0.755 & 1.358 & 1.151 \\
%     CSL & 1.194 & 1.465 & 1.582 & 1.829 & 2.249 \\
%     \bottomrule
%     \end{tabular}}
%     \label{tab:performance_comparison_2}
% \end{table}


% \begin{table}[h]
%     \centering
%     \caption{s/ep with varying length on EigenWorms dataset.}
%     \resizebox{.4\linewidth}{!}{\begin{tabular}{lcccc}
%     \toprule
%     Model & 300 & 600 & 900 & 1200 \\
%     \midrule
%     MMFA & 6.141 & 6.222 & 10.275 & 11.761 \\
%     TS2Vec & 2.791 & 2.886 & 3.823 & 4.122 \\
%     T-Loss & 9.150 & 10.601 & 9.954 & 9.299 \\
%     TS-TCC & 2.044 & 1.904 & 2.764 & 4.424 \\
%     TST & 0.413 & 0.514 & 1.073 & 1.038 \\
%     CSL & 1.059 & 1.275 & 1.573 & 1.678 \\
%     \bottomrule
%     \end{tabular}}
%     \label{tab:performance_comparison}
% \end{table}



\section{Proofs of Theories}

\subsection{\textbf{Theorem}~\ref{the:seman_sim}: Equivalence Between Eigenvalues and Distance Reduction of Spectral Embeddings.}\label{proof:eq}


\begin{proof}
    
We begin by expanding the expected squared difference:

\begin{equation}
\begin{aligned}
& \mathbb{E}_{(x, x^\prime)\sim p_{\text{\textit{sim}}}} [(f(x) - f(x^\prime))^2] \\
& = 2\mathbb{E}_{x \sim p_{\text{\textit{data}}}^\prime}[f(x)^2] - 2\mathbb{E}_{(x, x^\prime)\sim p_{\text{\textit{sim}}}}[f(x)f(x^\prime)] \\
&  \\ & \text{Expand the second term.} \\ \\
& =  2\mathbb{E}_{x \sim p_{\text{\textit{data}}}^\prime}[f(x)^2] \\ 
& \mathbb{E}_{x \sim p^\prime_{\text{\textit{data}}}, x^\prime \sim p_{\text{\textit{data}}}^\prime}[f(x) \int_{x^\prime} \frac { \frac 1 k p_{T_{(x)}, T_{(x^\prime)}}(x, x^\prime)} {p_{\text{\textit{data}}}^\prime(x)} f(x^\prime) dx^\prime] \\
&  \\ & \text{According to the eigenfunction property.} \\ \\
& = 2\mathbb{E}{x \sim p_{\text{\textit{data}}}^\prime}[f(x)^2] - \underbrace{2\mathbb{E}{x \sim p_{\text{\textit{data}}}^\prime}[(1-\lambda) f(x)^2]}_{\text{According to Eq~\ref{eq:eigen}.}} \\
& = 2\lambda\mathbb{E}_{x \sim p_{\text{\textit{data}}}^\prime}[f(x)^2]
\end{aligned}
\end{equation}

Thus, we have established the desired relationship:

\begin{equation}
\begin{aligned}
\mathbb{E}_{(x, x^\prime)\sim p_{\text{\textit{sim}}}} [(f(x) - f(x^\prime))^2]
& = 2\lambda\mathbb{E}{x \sim p_{\text{\textit{data}}}^\prime}[f(x)^2]
\end{aligned}
\end{equation}


\end{proof}


\subsection{\textbf{Theorem}~\ref{the:inv_est}: Multi-modal Invariance Estimation.}\label{proof:inv}

\begin{proof}
The proof of this theorem follows from the established inequalities detailed in Inequation~\ref{eq:invprime_prof}.

\begin{equation}
\begin{aligned}\label{eq:invprime_prof}
        \mathcal{L}_{inv}^\prime(Z) & = 
         \frac{1}{Nk(k + 1)}\sum_{i=1}^N(2\sum_{m=1}^k||z^{(0)}_i-z^{(m)}_i||^2_2 \\
        & + \sum_{\substack{m=1, n=1 \\ m \neq n}}^k||z^{(m)}_i - z^{(n)}_i||^2_2) \\
        & \leq \frac{1}{Nk(k + 1)}\sum_{i=1}^N[2\sum_{m=1}^k||z^{(0)}_i-z^{(m)}_i||^2_2 \\
        & + \sum_{\substack{m=1, n=1 \\ m \neq n}}^k(\underbrace{||z^{(0)}_i - z^{(m)}_i||^2_2+||z^{(0)}_i - z^{(n)}_i||^2_2)}_{\text{\textit{Triangle inequation.}}}] \\
        & = \frac{2}{N(k+1)} \sum_{i=1}^N\sum_{m=1}^k ||z^{(0)}_i - z^{(m)}_i||^2_2
\end{aligned}
\end{equation}
\end{proof}


\subsection{\textbf{Theorem}~\ref{the:orth}: Orthogonality of Eigenfunction-Based Representations.}

\begin{proof}



When we have the $n$ dimensional matrice $(D-w)$ which has finite rank, $||(D-w) - \mathbb{L}|| \xrightarrow{n \to \infty} 0 $ in an approximation sense. Thus, $\mathbb{L}$ is a compact operator. Therefor $\mathbb{L}=\sum_{i=1}^\infty \lambda_i \mathbb{E}_{x\sim p_{\text{data}}^\prime}[\cdot f_i(x)] f_i(x)$. For all $i \in [d_z] $, we have $\lambda_i \neq 0 $,  $ \mathbb{E}_{x \sim p_{\text{\textit{data}}}^\prime}[f_i(x)^2] = 1$. We prove that a set of $d_z$ eigenfunctions exists that are orthogonal to each other. The operator is symmetric.

\begin{equation}\label{eq:symetric}
    \mathbb{E}_{x \sim p_{\text{\textit{data}}}^\prime}[\mathbb{L}(f_i)(x) \cdot f_j(x)] = \mathbb{E}_{x \sim p_{\text{\textit{data}}}^\prime}[f_i(x) \cdot \mathbb{L}(f_j)(x)]
\end{equation}


According to the definition of eigenfunction, we have the equation below.

\begin{equation}\label{eq18}
    \lambda_i \mathbb{E}_{x \sim p_{\text{\textit{data}}}^\prime}[f_i(x) \cdot f_j(x)] =  \lambda_j \mathbb{E}_{x \sim p_{\text{\textit{data}}}^\prime}[f_i(x) \cdot f_j(x)]
\end{equation}

According to Eq.~\ref{eq18}, for any $\lambda_i \neq \lambda_j$ we have:
\begin{equation}
\begin{aligned}
     &   \forall i\neq j, \mathbb{E}_{x \sim p_{\text{\textit{data}}}^\prime}[f_i(x) \cdot f_j(x)] = 0
\end{aligned}
\end{equation}

When we have:

\begin{equation}
    \mathcal{S}_{\lambda} = \{f | \mathbb{L}(f)=\lambda f\} \ \ \text{dim}(\mathcal{S}_{\lambda}) \geq 2
\end{equation}

This generates a linear space.

\begin{equation}
\begin{aligned}
    & \mathbb{L}(ah + bg) = \mathbb{L}(ah) + \mathbb{L}(bg) = \lambda(ah + bg)  \\
    & \ \ \ \ \ \ \  a,b\in \mathbb{R} \ \ h,g \in \mathcal{S}_{\lambda} 
\end{aligned}
\end{equation}

We have $g_1 \in \mathcal{S}_\lambda$. Then, we remove the subspace spanned by $g_1$. Now we consider the space $\mathcal{S}^{\text{dim}(\mathcal{S}_{\lambda}) - 1} = \text{span}(\{g_1\})^\perp \cap \mathcal{S}_{\lambda}$. Then we have $g_i \in \mathcal{S}^{\text{dim}(\mathcal{S}_{\lambda}) - i + 1}$ where $\mathcal{S}^{\text{dim}(\mathcal{S}_{\lambda}) - i + 1}= \text{span}(\{g_j\}_{i-1})^\perp\cap \mathcal{S}_{\lambda}$. Finally, we have all the $\text{dim}(\mathcal{S}_{\lambda})$ orthogonal eigenfunctions with their eigenvalues equal to $\lambda$.

Ideally, we design an algorithm to learn $d_z$ of all the orthogonal eigenfunctions with the smallest eigenvalues for representations.

\end{proof}

\subsection{Linearly Degenerated MMFA (LD-MMFA) Performs as a Spectral Selection Mechanism to Capture Empirical Distributions}



This chapter establishes three foundational theorems for MMFA under linear degeneracy constraints. Each theorem provides a closed-form spectral solution that reconciles empirical distributions across transformed feature spaces while preserving orthogonality. Rigorous derivations and eigenvalue selection criteria are explicitly analyzed. We first briefly review the configuration constraints and training objectives of linear degenerated MMFA. Then we introduce the gradually deduced closed-form solutions for the simple to complex construction of the training objectives. 

\paragraph{Brief Introduction to Linear Degenerated MMFA and Unsupervised Objective.}\label{sec:recover}

In linearly degenerated MMFA, the core objective is to learn a linear projection \( Z = XW \) that maps standardized high-dimensional data \( X \in \mathbb{R}^{n \times d} \) into a low-dimensional latent space \( Z \in \mathbb{R}^{n \times k} \), while preserving critical statistical structures across multiple transformed feature distributions. This unsupervised framework operates without labeled data, relying instead on harmonizing empirical distributions measured from diverse feature transformations (e.g., \( F_1, F_2, F_3 \)) applied to \( X \). 


The optimization problem is designed jointly.
\underline{(1) Minimize Projection Complexity}: Penalize the Frobenius norm \( \|W\|_F^2 \) to select most significant spectrums to enhance generalizability.  
\underline{(2) Align Transformed Features}: Reduce discrepancies \( \|XW - F_i W_i'\|_F^2 \) between \( Z \) and each transformed view \( F_i W_i' \).  
\underline{(3) Enforce Orthogonality}: Constrain \( Z^TZ= W^\top X^\top X W = I_k \) and \( W_i'^\top F_i^\top F_i W_i' = I_k \) to ensure non-redundant, unit-covariance latent features.  

In the theorems below, we compose these objectives and constraints, deduce their close-form solutions, and find out each spectral selection mechanism.


    
\begin{lemma}\label{lem:PCA}LD-MMFA Without Transform Alignment Recovers PCA.

The solution recovers the scaled PCA’s principal directions. We use \( W^\top X^\top X W = I_k \) to forces orthogonality, but minimizing \( \|W\|_F^2 \) penalizes scaling inversely to variance. Thus, directions with maximal variance (largest \( \lambda_i \)) dominate, as they require minimal scaling to satisfy the orthogonality constraint.

The learning objective of LD-MMFA without transform alignment can be written as:

\begin{equation}
    \min_{W} \mathcal{L} = \|W\|_F^2 \quad \text{s.t.} \quad W^\top X^\top X W = I_k.
\end{equation}

Closed-form solution:  

\begin{equation}    
    W_{\text{opt}} = V_k \Lambda_k^{-1/2}, \quad \min_{W} \mathcal{L} = \sum_{i = 1}^k \lambda_{i}^{-1} \quad \text{s.t.} \quad W^\top X^\top X W = I_k.
\end{equation}

$\lambda_{X,i}$ is the $i$th largest singular value of $X^TX$.


\end{lemma}


\begin{proof}

To enforce the constraint \(W^\top X^\top X W = I_k\), we introduce a symmetric Lagrange multiplier matrix \(\Lambda \in \mathbb{R}^{k \times k}\) and construct the Lagrangian:

\begin{equation}
    \mathcal{L}(W, \Lambda) = \text{tr}(W^\top W) - \text{tr}\left(\Lambda^\top (W^\top X^\top X W - I_k)\right).    
\end{equation}

Here, the trace operator \(\text{tr}(\cdot)\) is used to reformulate the Frobenius norm and constraints into scalar forms suitable for differentiation.


Taking the derivative of \(\mathcal{L}\) with respect to \(W\) and setting it to zero yields:

\begin{equation}
\begin{aligned}
 & \frac{\partial \mathcal{L}}{\partial W} = 2W - 2 X^\top X W \Lambda = 0. \\
& \Rightarrow W = X^\top X W \Lambda. \\
& \Rightarrow W^\top W = W^\top X^\top X W \Lambda = I_k \Lambda . \\
& \Rightarrow  \Lambda = W^\top W. \\
& \Rightarrow   W = X^\top X W (W^\top W). \\
& \Rightarrow  X^\top X W = W (W^\top W)^{-1}.
\end{aligned}
\end{equation}



To interpret this, let \(W = [w_1, \dots, w_k]\) with columns \(w_i \in \mathbb{R}^d\). This implies that each column \(X^\top X w_i\) must lie within the column space of \(W\). Specifically, for each \(i\):
\begin{equation}
    X^\top X w_i = \sum_{j=1}^k c_{ji} w_j.    
\end{equation}
where \(c_{ji}\) are entries of the matrix \((W^\top W)^{-1}\). This indicates that the subspace \(\text{Span}(w_1, \dots, w_k)\) is invariant under the linear transformation defined by \(X^\top X\).

For symmetric matrices like \(X^\top X\), the Spectral \textbf{Theorem} guarantees that any invariant subspace is spanned by eigenvectors of \(X^\top X\). Therefore, the columns of \(W\) must be linear combinations of \(X^\top X\)'s eigenvectors.


Assume there exists another solution \( W' \), whose columns are composed of non-trivial combinations of multiple eigenvectors. According to the constraint \( W'^\top X^\top X W' = I_k \), the columns of \( W' \) must satisfy:  
\begin{equation}
w_i'^\top X^\top X w_j' = \delta_{ij}.
\end{equation}
If \( w_i' \) contains different eigenvectors, e.g., \( w_i' = a_{i1}v_1 + a_{i2}v_2 \) (where \( v_1, v_2 \) are eigenvectors of \( X^\top X \)), then:  
\begin{equation}
    w_i'^\top X^\top X w_j' = a_{i1}a_{j1}\lambda_1 + a_{i2}a_{j2}\lambda_2 = \delta_{ij}.
\end{equation}  
For \( i \neq j \), this requires:  
\begin{equation}
    a_{i1}a_{j1}\lambda_1 + a_{i2}a_{j2}\lambda_2 = 0.    
\end{equation}
This equation can only hold if \( a_{i1}a_{j1} = a_{i2}a_{j2} = 0 \). This forces \( w_i' \) and \( w_j' \) to exclusively contain single eigenvectors (i.e., either \( a_{i1} = a_{j1} = 0 \) or \( a_{i2} = a_{j2} = 0 \)). Consequently, the only feasible solution is one where each \( w_i' \) corresponds to a scaled version of a single eigenvector, and distinct columns correspond to distinct eigenvectors.  


Thus we decompose \( X^\top X \) into its spectral components: \( X^\top X = V \Lambda V^\top \), where \( V \) is an orthogonal matrix of eigenvectors, and \( \Lambda = \text{diag}(\lambda_1, \dots, \lambda_d) \) contains eigenvalues sorted as \( \lambda_1 \geq \lambda_2 \geq \cdots \geq \lambda_d \).  

We denote $Q$ as introduced parameters to analyze the spectral selection process, by expressing \( W \) as \( W = V Q \), the constraint simplifies to \( Q^\top \Lambda Q = I_k \). The objective \( \|W\|_F^2 = \text{tr}(Q^\top Q) \) then becomes equivalent to minimizing \( \text{tr}(Q^\top Q) \) under the diagonalized constraint \( Q^\top \Lambda Q = I_k \). To satisfy this, \( Q \) must take the form \( Q = \Lambda_k^{-1/2} \), where \( \Lambda_k \) is the submatrix of \( \Lambda \) corresponding to the largest \( k \) eigenvalues.  

Minimizing \( \text{tr}(Q^\top Q) = \text{tr}(\Lambda_k^{-1}) \) directly penalizes small eigenvalues. Selecting the largest eigenvalues \( \lambda_1, \dots, \lambda_k \) minimizes this penalty, as their reciprocals \( 1/\lambda_i \) are the smallest. Geometrically, directions with higher variance (larger \( \lambda_i \)) require less scaling to satisfy \( W^\top X^\top X W = I_k \), aligning with PCA’s variance-maximization principle.  

Closed-form solution:  
\begin{equation}
    W_{\text{opt}} = V_k \Lambda_k^{-1/2}, \quad \min_{W} \|W\|_F^2 = \sum_{i = 1}^k \lambda_{i}^{-1} \quad \text{s.t.} \quad W^\top X^\top X W = I_k.
\end{equation}  
where \( V_k \) are the top-\(k \) eigenvectors of \( X^\top X \), and \( \Lambda_k \) contains the corresponding eigenvalues. The solution recovers PCA’s principal directions scaled inversely by the square roots of their variances.  
    
\end{proof}
 

\begin{lemma}\label{lem:CCA}LD-MMFA Selects Spectral Consensus with Single Transformed Feature.


After solving LD-MMFA with single transformed feature $F = T(X)$, larger singular values \( \sigma_i \) of the whitened cross-covariance matrix $\bar{X}^T \bar{F} = (X V_X\Lambda_X^{-1/2})^TFV_F\Lambda_F^{-1/2}$ signify cross-view alignment. The solution projects \( X \) and \( F \) onto directions of maximal cross-view correlation, scaled by their respective whitening matrices. This balances regularization penalties while aligning the transformed features.  


The learning objective of LD-MMFA with single transform alignment can be written as:

\begin{equation}
    \min_{W, W'} \mathcal{L} = \|W\|_F^2 + \|W'\|_F^2 + \|XW - FW'\|_F^2 \quad \text{s.t.} \quad W^\top X^\top X W = W'^\top F^\top F W' = I_k.
\end{equation}  

Closed-Form Solution:  
\begin{equation}
\begin{aligned}
    & W_{\text{opt}} = V_X\Lambda_X^{-1/2} U_k, \quad W'_{\text{opt}} = V_F\Lambda_F^{-1/2} V_k, \\
    & \min_{W, W'} \mathcal{L} = \sum_{i=1}^{k} ( \frac{1}{\lambda_{X,i}} + \frac{1}{\lambda_{F,i}} - 2 \sigma_i ) + 2k \quad \text{s.t.} \quad W^\top X^\top X W = W'^\top F^\top F W' = I_k.
\end{aligned}
\end{equation}  


where \( U_k, V_k \) are the top-\(k \) singular vectors of \( C = V_X\Lambda_X^{-1/2} X^\top F V_F\Lambda_F^{-1/2} \). And $\lambda_{X,i}, \lambda_{F,i}$ and $\sigma_i$ are the $i$th largest singular value of $X^TX, F'^\top F'$ and $\bar{X}^T \bar{F}$.

This optimization problem recovers canonical correlation analysis (CCA)~\cite{thompson2000canonical} for LD-MMFA and deep/kernel canonical correlation analysis (KCCA)~\cite{fukumizu2007statistical} for MMFA.

\end{lemma}

\begin{proof}
    

The constraints \( W^\top X^\top X W = I_k \) and \( W'^\top F^\top F W' = I_k \) suggest whitening both \( X \) and \( F \). Define the whitened matrices \( \tilde{X} = X V_X\Lambda_X^{-1/2} \) and \( \tilde{F} = F V_F\Lambda_F^{-1/2} \), which satisfy \( \tilde{X}^\top \tilde{X} = I \) and \( \tilde{F}^\top \tilde{F} = I \). 

We still denote $Q$ as the mapping for spectral selection. Substituting \( W = V_X\Lambda_X^{-1/2} Q \) and \( W' = V_F\Lambda_F^{-1/2} Q' \), the constraints reduce to \( Q^\top Q = I_k \) and \( Q'^\top Q' = I_k \).  

The objective then becomes:  
\begin{equation}
\begin{aligned}
& \| \tilde{X} Q - \tilde{F} Q' \|_F^2 + \text{tr}(Q^\top \Lambda_X^{-1} Q) + \text{tr}(Q'^\top \Lambda_F^{-1} Q') \\
&  = \text{tr}\left(Q^\top \Lambda_X^{-1} Q\right) + \text{tr}\left(Q'^\top \Lambda_F^{-1} Q'\right) - 2 \text{tr}(Q^\top C Q') + 2k.
\end{aligned}
\end{equation}  

Expanding the squared Frobenius norm and ignoring constant terms, this simplifies to maximizing \( \text{tr}(Q^\top \tilde{X}^\top \tilde{F} Q') \). The cross-term \( \tilde{X}^\top \tilde{F} \) defines the whitened cross-covariance matrix \( C = \tilde{X}^\top \tilde{F} \).  

To maximize \( \text{tr}(Q^\top C Q') \), we apply the Singular Value Decomposition (SVD) to \( C \), yielding \( C = U \Sigma V^\top \), where \( U \) and \( V \) are orthogonal matrices, and \( \Sigma \) contains singular values \( \sigma_1 \geq \sigma_2 \geq \cdots \geq \sigma_{\min(d,m)} \). The optimal \( Q \) and \( Q' \) are the first \( k \) columns of \( U \) and \( V \), respectively.  


The trace \( \text{tr}(Q^\top C Q') = \sum_{i=1}^k \sigma_i \) is maximized when \( Q \) and \( Q' \) align with the singular vectors corresponding to the largest \( \sigma_i \). Larger singular values represent stronger correlations between the whitened \( X \) and \( F \), minimizing the residual \( \|XW - FW'\|_F^2 \).  

Closed-Form Solution:  
\begin{equation}
\begin{aligned}
& W_{\text{opt}} = V_X\Lambda_X^{-1/2} U_k, \quad W'_{\text{opt}} = V_F\Lambda_F^{-1/2} V_k, \\
& \min_{W, W'} \mathcal{L} = \sum_{i=1}^{k} ( \frac{1}{\lambda_{X,i}} + \frac{1}{\lambda_{F,i}} - 2 \sigma_i ) + 2k \quad \text{s.t.} \quad W^\top X^\top X W = W'^\top F^\top F W' = I_k.
\end{aligned}
\end{equation}  

where \( U_k, V_k \) are the top-\(k \) singular vectors of \( C = V_X\Lambda_X^{-1/2} X^\top F V_F\Lambda_F^{-1/2} \). The solution projects \( X \) and \( F \) onto directions of maximal cross-view correlation, scaled by their respective whitening matrices. This balances regularization penalties while aligning the transformed features.  
\end{proof}

    
\begin{lemma}\label{lem:multiCCA}
Multiview Alignment via Joint Cross-Covariance Spectrum.
The concatenated matrix \( C \) encodes all pairwise correlations between \( X \) and each \( F_i \). Its SVD extracts a consensus subspace \( U_k \) in \( X \) that maximizes the total correlation with all \( F_i \)-views. Each \( V_{i,k} \) aligns \( F_i \) to this subspace. Larger singular values prioritize directions where \( X \) simultaneously explains multiple \( F_i \). The solution minimizes the total residual across all views while satisfying per-view orthogonality constraints.

Optimization Problem:  
\begin{equation}
\begin{aligned}
\min_{W, \{W_i'\}} \mathcal{L} = \|W\|_F^2 + \sum_{i=1}^p \left( \|W_i'\|_F^2 + \|XW - F_i W_i'\|_F^2 \right) \quad \text{s.t.} \quad W^\top X^\top X W = W_i'^\top F_i^\top F_i W_i' = I_k.
\end{aligned}
\end{equation}  
Closed-Form Solution:  
\begin{equation}
\begin{aligned}
& W_{\text{opt}} = V_X\Lambda_X^{-1/2} U_k, \quad W_{i,\text{opt}}' = V_{F_i}\Lambda_{F_i}^{-1/2} V_{i,k}, \\
& \min_{W, W'} \mathcal{L} = \sum_{j=1}^{k} ( \frac{1}{\lambda_{X,j}} + \sum_{i=0}^p \frac{1}{\lambda_{F,i,j}} - 2 \sigma_j ) + 2k \quad \text{s.t.} \quad W^\top X^\top X W = W_i'^\top F_i^\top F_i W_i' = I_k.
\end{aligned}
\end{equation}  

$\lambda_{X,j}, \lambda_{F,i,j}$ and $\sigma_j$ are the $j$th largest singular value of $X^TX, F_i'^\top F_i'$ and $\bar{X}^T \bar{F}$.

\end{lemma}

\begin{proof}


where \( U_k \) and \( V_{i,k} \) are derived from the SVD of \( C = [\tilde{X}^\top \tilde{F}_1 \; \cdots \; \tilde{X}^\top \tilde{F}_p] \).

Derivation:  
Extending \textbf{Theorem} 2 to multiple views, we first whiten \( X \) and each \( F_i \):  
\begin{equation}
\begin{aligned}
\tilde{X} = X V_X\Lambda_X^{-1/2}, \quad \tilde{F}_i = F_i V_{F_i}\Lambda_{F_i}^{-1/2}.
\end{aligned}
\end{equation} 
Substituting \( W = V_X\Lambda_X^{-1/2} Q \) and \( W_i' = V_{F_i}\Lambda_{F_i}^{-1/2} Q_i' \), the constraints reduce to \( Q^\top Q = I_k \) and \( Q_i'^\top Q_i' = I_k \).  

The objective simplifies to maximizing \( \sum_{i=1}^p \text{tr}(Q^\top \tilde{X}^\top \tilde{F}_i Q_i') \), which aggregates cross-view correlations. This is equivalent to maximizing \( \text{tr}(Q^\top C [Q_1'^\top \; \cdots \; Q_p'^\top]^\top) \), where \( C = [\tilde{X}^\top \tilde{F}_1 \; \cdots \; \tilde{X}^\top \tilde{F}_p] \).  

Applying SVD to \( C \), we write \( C = U \Sigma [V_1^\top \; \cdots \; V_p^\top]^\top \), where \( U \) and each \( V_i \) are orthogonal. The optimal \( Q \) is the first \( k \) columns of \( U \), while each \( Q_i' \) corresponds to the first \( k \) columns of \( V_i \).  


The concatenated matrix \( C \) captures all pairwise correlations between \( X \) and the transformed views \( F_i \). The SVD of \( C \) extracts a shared subspace in \( X \) (spanned by \( U_k \)) that maximizes the total correlation with all \( F_i \)-views. Each \( V_{i,k} \) aligns \( F_i \) to this subspace. Larger singular values prioritize directions where \( X \) simultaneously explains multiple \( F_i \).  

The solution harmonizes multiple transformed views by projecting them onto a consensus subspace in \( X \), weighted by their joint cross-covariance spectrum. This minimizes the total residual across all views while preserving orthogonality constraints.  


\end{proof}

\subsection{Theorem~\ref{theorem:recover}: MMFA's recovery of KPCA and KCCA.}\label{sec:recoverk}

The following proof demonstrates that under linear degeneracy constraints, MMFA recovers Kernel Principal Component Analysis (KPCA)~\cite{kpca} and Kernel Canonical Correlation Analysis (KCCA)~\cite{fukumizu2007statistical} when applied to kernel-mapped data. The linear solutions derived in \textbf{Lemmas} \ref{lem:PCA}–\ref{lem:multiCCA} generalize to their kernelized counterparts through the kernel trick, where input data \(X\) and transformed features \(F_i\) are implicitly mapped into reproducing kernel Hilbert spaces (RKHS).


\begin{proof}
When \(p = 0\) (no transformed features), we apply LD-MMFA to kernel-mapped data \(\Phi_X(X)W\) with \(p = 0\). According to \textbf{Lemma}~\ref{lem:PCA}, the solution to LD-MMFA reduces to KPCA,

When aligning a single transformed feature \(\Psi_F(F)W'\) (kernel-mapped from another view \(F\)), we apply LD-MMFA to kernel-mapped data \(\Phi_X(X)W\) and \(\Phi_F(F)W'\) with \(p = 1\). According to \textbf{Lemma}~\ref{lem:CCA} the solution recovers KCCA, selecting directions of maximal correlation between kernel spaces.

For multiple transformed features \(\{\Psi_{F,i}(F_i)W_{F,i}\}\), the solution aligns a consensus subspace across all kernel-induced views, according to \textbf{Lemma}~\ref{lem:multiCCA} generalizing multi-view KCCA.
\end{proof}

%%%%%%%%%%%%%%%%%%%%%%%%%%%%%%%%%%%%%%%%%%%%%%%%%%%%%%%%%%%%%%%%%%%%%%%%%%%%%%%
%%%%%%%%%%%%%%%%%%%%%%%%%%%%%%%%%%%%%%%%%%%%%%%%%%%%%%%%%%%%%%%%%%%%%%%%%%%%%%%


\end{document}


% This document was modified from the file originally made available by
% Pat Langley and Andrea Danyluk for ICML-2K. This version was created
% by Iain Murray in 2018, and modified by Alexandre Bouchard in
% 2019 and 2021 and by Csaba Szepesvari, Gang Niu and Sivan Sabato in 2022.
% Modified again in 2023 and 2024 by Sivan Sabato and Jonathan Scarlett.
% Previous contributors include Dan Roy, Lise Getoor and Tobias
% Scheffer, which was slightly modified from the 2010 version by
% Thorsten Joachims & Johannes Fuernkranz, slightly modified from the
% 2009 version by Kiri Wagstaff and Sam Roweis's 2008 version, which is
% slightly modified from Prasad Tadepalli's 2007 version which is a
% lightly changed version of the previous year's version by Andrew
% Moore, which was in turn edited from those of Kristian Kersting and
% Codrina Lauth. Alex Smola contributed to the algorithmic style files.
