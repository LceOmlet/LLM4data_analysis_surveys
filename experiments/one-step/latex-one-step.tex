\documentclass{article}
\usepackage[utf8]{inputenc}
\usepackage{geometry}
\usepackage{hyperref}
\usepackage{amsmath}
\usepackage{amssymb}
\usepackage{booktabs}  % For three-line table
\usepackage{array}     % For custom column types
\usepackage{graphicx}  % For resizebox
\usepackage{multirow} % For multirow tables

\begin{document}


The study evaluates 33 models—including large language models (LLMs), supervised algorithms, and unsupervised baselines—on synthetic regression benchmarks designed to test their ability to infer linear and non-linear relationships. All datasets are synthetically generated with controlled distributions to ensure reproducibility and avoid pre-exposure to LLMs during pre-training. 

The linear tasks focus on sparse feature identification, such as datasets where only 1 out of 3 features is informative (e.g., \( y = \beta x_1 + \epsilon \), with \( x_2, x_3 \) as noise). For non-linear tasks, the benchmarks include classical functions like Friedman\#1 (\( y = 10\sin(\pi x_0x_1) + 20(x_2-0.5)^2 + 10x_3 + 5x_4 + \epsilon \)) and custom-designed equations such as Original\#1 (\( y = x + 10\sin(5\pi x/100) + 10\cos(6\pi x/100) \)), which combines linear trends with periodic oscillations.  

Tab~\ref{tab:one-step-performance} which is summarized ranking results of, reveal that LLMs (e.g., Claude 3 Opus, GPT-4) achieve strong performance across tasks by leveraging in-context learning (ICL), even outperforming specialized supervised methods like Gradient Boosting in non-linear scenarios. For instance, Claude 3 Opus excels in approximating complex functional forms (e.g., Friedman\#2: \( y = \sqrt{x_1^2 + (x_2x_3 - 1/(x_2x_4))^2} + \epsilon \)) without gradient updates, though it occasionally struggles with extreme outliers. Supervised models like deep MLPs dominate sparse linear tasks but falter on non-linear interactions, while Linear Regression with Polynomial Features shows competitive performance on certain synthetic curves. Unsupervised baselines (e.g., random guessing) consistently underperform, highlighting the sophistication of ICL.  



\begin{table*}
    \centering
    \caption{\textbf{Performance rankings of 10 out of 33 models on linear and non-linear tasks.} The table shows the ranking performance of LLMs, supervised models, and unsupervised baselines across different task types. Lower ranks indicate better performance.}
    \label{tab:one-step-performance}
    \resizebox{1\textwidth}{!}{
    \begin{tabular}{l l c c c}
        \toprule
        \textbf{Model Category} & \textbf{Model Name} & \textbf{Linear Tasks} & \textbf{Non-linear Tasks} & \textbf{Overall Ranking} \\
        \midrule
        \multirow{3}{*}{\textbf{LLMs}} & Claude 3 Opus & 4.75 & 3.0 & 3.88 \\
        & GPT-4 & 11.5 & 4.4 & 7.95 \\
        & DBRX & 20.25 & 16.2 & 18.23 \\
        \cmidrule{1-5}
        \multirow{5}{*}{\textbf{Supervised Models}} & MLP Deep 2 & 1.25 & 20.2 & 10.73 \\
        & MLP Deep 3 & 2.5 & 18.8 & 10.65 \\
        & Gradient Boosting & 16.5 & 8.0 & 12.25 \\
        & Linear Reg + Poly & 27.0 & 18.0 & 22.5 \\
        & Random Forest & 5.0 & 21.4 & 13.2 \\
        \cmidrule{1-5}
        \multirow{2}{*}{\textbf{Unsupervised Baselines}} & Average & 30.25 & 22.0 & 26.13 \\
        & Random & 13.5 & 34.0 & 23.75 \\
        \bottomrule
    \end{tabular}}
\end{table*}

\end{document}